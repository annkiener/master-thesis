\chapter{Conclusion}
\label{ch:conclusions}


This master thesis investigated the effectiveness of a brief Mixed Reality simulation designed to enhance both affective and cognitive empathy in medical students towards patients with schizophrenia. The aim was to explore whether immersing students in simulated symptoms, complemented by education and structured debriefing, could reshape their empathy and perceptions towards people living with this stigmatized and often misunderstod condition.

\section{Key Findings}
While the quantitative analysis of this study, using the JSE and B-PANAS, did not show statistically significant changes in empathy scores or emotional affect across the whole group, important insights were gained, even from these results. The pre-evaluation revealed that participants already possessed a relatively high baseline level of empathy, which may have limited the room for significant improvement.

\vspace{1em}

Addtionally, the qualitative findings provided evidence of meaningful cognitive and emotional engagement. Participants experiencing the simulation frequently reported strong emotional reactions, including difficulties in focusing, feelings of helplessness, and general discomfort. Crucially, many participants articulated that the experience altered their perspective on living with schizophrenia, showing increased compassion and understanding. There was also some variability in how visible users reactions were to the observing participants, which shows different nature of the immersive experience. The auditory hallucinations were often perceived as overwhelming and possibly the most impactful aspect of the simulation.


\section{Impact}
This thesis adds to the field by focusing on Mixed Reality as a tool for empathy training in medical education, addressing a gap in existing research which is more centered on Virtual Reality. MR offers unique advantages by enabling users to experience simulated symptoms while remaining in their real-world environment, therefore promoting emotional safety and relatability compared to fully immersive VR experiences which can sometimes be overwhelming.

The successful development and testing of a MR application, which can repeatedly be used, demonstrates its potential as a valuable tool for traditional learning methods in medical studies. This project highlights the capability of MR to offer an authentic yet safe experiential learning environment. In addition to that, it has the potential to prepare future healthcare providers for more compassionate and understanding interactions with patients suffering from schizophrenia or other mental health conditions. Working closely with healthcare experts and following strong ethical rules also made this project more scientifically sound and valuable for education.


\section{Outlook}

A key takeaway from this project is how crucial a good learning setup is. This includes preparing students before the simulation and, most importantly, discussing it with them afterward. Even if the measurements do not show big changes, these guided talks are crucial. They help students process the experience, understand it better, and turn their emotional reactions into empathy. This also helps prevent increasing negative ideas or stigma or making students feel more uncomfortable.

\paragraph{Future Work}
Based on what was found, here are some ideas for future research and improvements:

\begin{itemize}
    \item \textbf{Interaction Design:} The simulation could be more engaging. Some students felt unsure how to interact with the virtual parts. Giving clearer instructions or tasks that need direct responses within the Mixed Reality environment could make it more immersive and help students participate more.
    \item \textbf{Engagement:} There should be more emphasis into ways to make the experience better for students who are watching but not wearing the headset. This could help the whole group learn and develop empathy even more.
    \item \textbf{Long-term Effects and more Participants:} Future studies should check how these MR simulations affect empathy and attitudes over a longer period. In addition, it should also include more people in these studies to see if these findings apply more widely and if one can find bigger statistical differences.
    \item \textbf{Content:} The simulation content can be improved to adjust to how each user reacts. This would help find the right balance between how intense the experience is emotionally and how safe and comfortable the person feels. This way, more people can benefit from the experience.
\end{itemize}

In short, this study successfully showed that a new Mixed Reality application can be a powerful learning tool. It helps medical students better understand and empathize with people who have schizophrenia. Even though the numbers did not show major changes in empathy, the strong emotional and mental impact, plus the benefits of MR, prove its potential. This technology could really change mental health education and lead to more caring healthcare for everyone.