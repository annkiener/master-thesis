\chapter{User Study}
\label{ch:userstudy}


To systematically evaluate the educational and empathic impact of the MR simulation, a controlled user study is conducted with a total of 29 medical students enrolled at the University of Health in Fribourg, Switzerland (in French: Haute école de santé Fribourg, HEdS-FR). The study employed a pretest–posttest mixed-methods design, combining quantitative measures with group interaction and reflective feedback to understand the effects of the simulation experience.

\section{Study Design Overview}

The user study followed a five-phase structure: pre-simulation evaluation, simulation experience, post-simulation evaluation, debriefing session and data analysis (quantitative and qualitative evaluation). This structure allowed for a measurement of change in empathy and emotional response, and offered insight into both the direct effects of the MR simulation on the headset user and the indirect effects on observing students. The design aligns with educational best practices and recommendations in empathy training, particularly those emphasizing immersive realism combined with ethical framing and debriefing.

\vspace{1em}
The study was conducted in a controlled environment, with all participants receiving the same information and instructions. The simulation is designed to be brief yet impactful, allowing for a focused exploration of the experience of psychosis while minimizing potential distress.
The study is approved by the HEdS-FR ethics committee, and all participants provided informed consent prior to their involvement. The study was conducted in accordance with ethical guidelines for research involving human subjects, particularly in the context of medical education and simulation.

\vspace{1em}

The evaluation included both quantitative and qualitative components:

\begin{itemize}
  \item \textbf{Pre-Evaluation:} Conducted immediately before the simulation, this questionnaire assessed participants prior experience with patients (especially those diagnosed with schizophrenia), and measured baseline empathy and emotional perceptions.
  \item \textbf{Post-Evaluation:} Completed directly after the simulation, this questionnaire repeated the empathy and emotion assessments and included additional questions about the participants experience with the simulation.
\end{itemize}

Two groups were compared in the post-evaluation: participants who engaged in the simulation as "observers" (without the headset) and those who used the mixed-reality headset individually.

\section{Participants}

The target group for this study consists of medical students (n = 29) in their early clinical training, specifically from the HEdS-FR. This group of participants was selected for two primary reasons. First, students at this stage are actively developing their clinical attitudes, as they are still in the early stages of their career, including their capacity for empathy toward patients. Second, previous research has shown that empathy training tends to be particularly effective during this formative period in a healthcare professionals education \cite{Hsia2022, Kuhail2022}.

\vspace{1em}

Participation in the study is voluntary, and all participants are recruited through internal communication channels within the university by their professor. Before taking part, each participant receives comprehensive information about the objectives of the study, its procedures, and potential risks. They are informed of their rights, including the ability to withdraw at any time, and are asked to sign a written consent form confirming their understanding and agreement.


\section{Procedure}

The study is conducted in small groups. A total of five groups, each consisting of six students, participate in the simulation sessions. Within each group, only one student wears the MR headset and experiences the simulated symptoms. The other five students remain in the room during the simulation and are given a specific task by the instructor. Their role is to observe the behavior of the participant wearing the headset, noting any signs of confusion, distraction, or distress. This setup serves two purposes: first, it mirrors real scenarios where healthcare providers must interpret subtle behavioral cues; and second, it allows researchers to explore whether witnessing someone elses simulated experience can also affect empathy and perception from an external perspective.

\vspace{1em}

All six group members—both the headset user and the observers—complete the same set of questionnaires. These include the Jefferson Scale of Empathy (JSE) \cite{Hojat2002} to assess baseline and post-simulation empathy levels, and the Brief Positive and Negative Affect Schedule (B-PANAS) \cite{Boiroux2024} to measure emotional responses and perceptions toward individuals with schizophrenia. The evaluation process is described in more detail in Chapter~\ref{ch:eval}.

The simulation itself lasts approximately 3 to 4 minutes. During this time, the student wearing the headset is exposed to a carefully sequenced combination of auditory and visual hallucinations, all set within a familiar environment, which is the classroom. The goal is to simulate psychotic symptoms in a way that is immersive but safe, and to encourage emotional and cognitive engagement with the experience.

\vspace{1em}

Immediately following the simulation, participants once again complete the JSE and B-PANAS questionnaires to assess any changes in empathy levels and emotional responses. They are also invited to provide qualitative feedback on the simulation, including comments on its realism, emotional impact, and educational value. The inclusion of both direct and indirect participants allows the study to assess how empathy might be influenced not only by immersive first-person experiences, but also through empathetic observation—a dimension that has received limited attention in the literature.

\vspace{1em}

After the post-evaluation, all group members take part in a structured debriefing session moderated by teaching staff. This guided reflection allows participants to discuss what they observed or experienced, process their emotional responses, and relate the exercise to their future clinical work. For the observers in particular, this provides an opportunity to articulate how witnessing the simulation affected their perception of both the symptoms and the individual undergoing them.

\begin{figure}[H]
\centering
\resizebox{\textwidth}{!}{%
\begin{tikzpicture}[node distance=1.4cm and 2.1cm,
  every node/.style={font=\large},
  process/.style={rectangle, draw=blue!70!black, fill=blue!10, rounded corners,
                  minimum height=4cm, text width=3.2cm, align=center},
  observer/.style={rectangle, draw=orange!90!black, fill=orange!10, rounded corners,
                   minimum height=4cm, text width=3.2cm, align=center},
  arrow/.style={thick, ->, >=Stealth}
]

\node[process] (pretest) {\textbf{Pre-Evaluation}\\[2pt] (Empathy and emotional baseline using JSE and B-PANAS)};
\node[process, right=of pretest] (simulation) {\textbf{MR Simulation}\\[2pt] (One headset user, five observers)};
\node[process, right=of simulation] (posttest) {\textbf{Post-Evaluation}\\[2pt] (Same measures + simulation perception)};
\node[process, right=of posttest] (debrief) {\textbf{Debriefing Session}\\[2pt] (Moderated group reflection)};
\node[observer, right=of debrief] (analysis) {\textbf{Data Analysis}\\[2pt] (Quantitative and qualitative evaluation)};

\draw[arrow] (pretest) -- (simulation);
\draw[arrow] (simulation) -- (posttest);
\draw[arrow] (posttest) -- (debrief);
\draw[arrow] (debrief) -- (analysis);

\end{tikzpicture}
}
\caption{Overview of the user study procedure}
\label{fig:userstudy_flowchart}
\end{figure}

An overview of this procedure is illustrated in Figure~\ref{fig:userstudy_flowchart}, which summarizes the five main phases of the study, from pre-evaluation through simulation and group reflection to final data analysis. This structured flow ensures both consistency across groups and comprehensive capture of quantitative and qualitative data.

\paragraph{Task Description}
During the simulation, all participants, including the headset user and the observers, were given a specific task to complete. This task was introduced with the instruction by the teacher: \textit{"Take a piece of paper. I am going to give you a short instruction. This is a small exercise in personal reflection. You do not need to write a long text, just a few ideas, words, or phrases. At the end, I will ask you a question related to what you have written down."}

The detailed instruction for the task then followed: \textit{"Take a moment to think about a time in your life when you felt proud of yourself. This can be a personal, academic, family, or professional event... Write down: What you did; What it taught you about yourself; A word or phrase that summarizes this memory."} 

This exercise was designed to provide a common, engaging activity for all participants while the simulation was going on.

\subsection{JSE and B-PANAS Scales}
\label{ch:eval}
The evaluation of the MR simulation's impact on empathy and emotional response is conducted using two primary measurement tools: the Jefferson Scale of Empathy (JSE) and the Brief Positive and Negative Affect Schedule (B-PANAS). These tools are designed to capture both cognitive and affective dimensions of empathy, as well as emotional responses to individuals with schizophrenia.

\subsubsection{Jefferson Scale of Empathy (JSE)}
\label{sec:jse}

The primary tool used to measure empathy is the Jefferson Scale of Empathy (JSE), which is widely applied in medical education and has been shown to reliably measure both affective and cognitive components of empathy \cite{Hojat2002}. The JSE is administered before and after the MR simulation to assess whether the experience has led to measurable changes in students’ empathy levels. The results are analyzed to determine changes in total empathy scores.

Since the JSE is originally developed in English and no officially validated French version is available for this study, the questionnaire is translated into French by the researcher using a combination of online translation tools and manual adjustments. While care is taken to preserve the meaning and intent of the original items, this translated version did not go through formal validation. As such, the use of this adapted French version represents a methodological limitation and should be considered when interpreting the results.

\vspace{1em}

To better align the measurement tool with the goals of this study, which are to evaluate both cognitive and affective components of empathy in a time-sensitive way—the full JSE is thematically reviewed and categorized by the author. Based on the literature review and the definitions of empathy used in this thesis, each item is classified as either \textit{Cognitive} or \textit{Affective}. Cognitive items reflect an emphasis on understanding the patient’s perspective, thoughts, or non-verbal cues, while affective items relate to emotional awareness, resonance, or the therapeutic value of emotional understanding. A detailed overview of this classification can be found in Appendix~\ref{app:jse}, Table~\ref{tab:jse_classification}.

In order to maintain engagement, a shortened version of the JSE is developed. This version includes 13 items—five reflecting cognitive empathy and 8 reflecting affective empathy—that were selected based on thematic clarity and their alignment with the measurement goals of the study. The item selection are shown in Appendix~\ref{app:jse-short}, Table~\ref{tab:jse_shortened}. % explain why these items?

\subsubsection{Emotional Response (Positive and Negative Affect)}

To better understand the emotional impact of the simulation, students were asked to rate the intensity of their own emotional responses when thinking specifically about individuals diagnosed with schizophrenia, to assess their perceptions of people with this disorder. This part of the questionnaire is adapted from a validated French-language version of the Positive and Negative Affect Schedule (PANAS), as published by Boiroux \cite{Boiroux2024}. Participants rated each emotion on a 5-point Likert scale ranging from 1 (“Pas du tout” / “Not at all”) to 5 (“Extrêmement” / “Extremely”).

\vspace{1em}

The selection of emotional terms is selected to include an equal balance of five positive and five negative affective states. The goal of this design is to explore how the simulation might shift students emotional associations with schizophrenia - either increasing compassionate or empathetic responses, or reducing feelings of fear, anxiety, or social discomfort. Rather than only recording whether emotions intensified or weakened overall, the approach focused on identifying which specific emotional tones were affected and in what direction.

The following ten emotions were included in the questionnaire:

\begin{quote}
\textit{Angoissé(e) (Anxious), Enthousiaste (Enthusiastic), Honteux(se) (Ashamed), Inspiré(e) (Inspired), Intéressé(e) (Interested), Irrité(e) (Irritated), Craintif(ve) (Fearful), Alerte (Alert), Attentif(ve) (Attentive), and Nerveux(se) (Nervous).}
\end{quote}

This set provides a balanced perspective on affective response. Positive terms such as \textit{enthousiaste}, \textit{inspiré(e)}, and \textit{intéressé(e)} were selected to assess potential increases in empathy and engagements following the simulation. On the other hand, negative emotions like \textit{angoissé(e)}, \textit{honteux(se)}, and \textit{craintif(ve)} were included to evaluate whether the experience reduced discomfort or fear.

\vspace{1em}

This measurement strategy supports a more nuanced understanding of how the MR simulation influenced the emotional lens through which students perceive individuals with schizophrenia. It complements the cognitive and affective empathy data from the JSE by offering insight into the emotional tone behind students attitudes.

\subsection{Perceptions of the Simulation}

In addition to the JSE and the emotional response, participants which wore the headset, complete a short questionnaire immediately after the simulation, which evaluates their perceptions of the experience. This includes five statements rated on a 7-point Likert scale (1 = “Strongly disagree” to 7 = “Strongly agree”). The items are designed to assess how educational, immersive, and useful the simulation is perceived to be, as well as its potential to increase understanding and empathy. These items are as follows:

\begin{itemize}
    \item La simulation était éducative. \\
    \textit{The simulation was educational.}

    \item La simulation est un moyen efficace de sensibiliser à la schizophrénie. \\
    \textit{The simulation is an effective way to raise awareness about schizophrenia.}

    \item La simulation ne doit pas être utilisée par les personnes qui souhaitent travailler avec des personnes atteintes de schizophrénie. \\
    \textit{The simulation should not be used by people who want to work with individuals with schizophrenia.}

    \item La simulation devrait rendre les gens plus compréhensifs à l’égard des personnes atteintes de schizophrénie. \\
    \textit{The simulation should help people become more understanding toward individuals with schizophrenia.}

    \item La simulation était très immersive et donnait vraiment l’impression d’être réelle. \\
    \textit{The simulation was very immersive and really felt real.}
\end{itemize}

This helps evaluate how participants interpreted the experience and whether they found it meaningful in a learning context.

\subsection{Debriefing and Reflection}

Following the simulation and post-questionnaire phase, all participants engaged in a structured debriefing session facilitated by their professor. This session provided a safe space for emotional and intellectual reflection. Participants were encouraged to share their thoughts, feelings, and interpretations of both the simulation and their observations. The discussion also addressed how the simulation might influence their attitudes or behaviors in future clinical interactions with patients diagnosed with schizophrenia.

\section{Data Collection and Analysis}

The data collected during the user study included both quantitative and qualitative components. Quantitative measures consisted of pre- and post-simulation scores on the JSE and B-PANAS for all participants, as well as Likert-scale responses to the simulation perception questionnaire completed by the headset user. These data were analyzed using statistical comparisons, to detect significant changes in empathy and emotional affect. 

\vspace{1em}

Qualitative data were derived from transcripts of the debriefing sessions. These responses were analyzed using themes which re-occurred to identify motifs, such as emotional resonance, perception of the 'realness', educational value, ethical reflections, and shifts in attitudes or understanding.

\section{Ethical Considerations}

Given the potentially distressing nature of psychosis simulation, the study is designed with multiple ethical elemnets. Participants received clear pre-study briefings and signed informed consent forms. The simulation is kept short in duration and kept in a familiar environment to minimize psychological risk. The debriefing session served not only as a pedagogical tool but also as a psychological buffer, ensuring that students could process the experience in a safe, reflective manner. Participants were also reminded of their right to withdraw at any point without consequence.
