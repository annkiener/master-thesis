\chapter{User Study}
\label{ch:userstudy}

\emph{In this section, you should present your user study and its methodology. }

\section{Methodology}

% This study employed a within-subjects mixed-method design to investigate the impact of a mixed reality (MR) simulation on cognitive and affective empathy in medical students. The experimental component involved participants experiencing a short, immersive simulation of schizophrenia symptoms using a mixed reality headset, while the control aspect involved observation and related tasks by non-headset participants. Empathy levels were measured using standardized questionnaires administered before and after the simulation. Participants were recruited from the university of health in Fribourg, Switzerland (fr. Haute école de santé Fribourg (HEdS-FR)). Inclusion criteria required participants to be currently enrolled medical students fluent in French or German and able to provide informed consent. All participants signed a consent form that clearly outlined the purpose, procedure, risks, and their rights, including the option to withdraw at any point. Participants were assured of confidentiality and anonymity per Swiss data protection regulations.

% The study utilized a mixed reality headset, namely the Meta Quest 3, with a backup unit available in case of technical difficulties. The core of the experimental procedure was a custom-built mixed reality simulation designed to replicate common positive symptoms of schizophrenia, including auditory hallucinations and perceptual disturbances. Each simulation lasted approximately three to five minutes and was rebuilt after every session to ensure consistent functionality. Data collection was facilitated through Microsoft Forms questionnaires. The pre-questionnaire gathered demographic information and assessed baseline levels of empathy using a validated empathy scale, such as the Jefferson Scale of Empathy or the Interpersonal Reactivity Index. Following the simulation, the post-questionnaire re-administered the empathy assessment and included open-ended items to capture participants’ qualitative reflections on the experience. Additional materials included printed consent forms for each participant, a list documenting participant names and group allocations, and basic administrative supplies such as pens, which were used for signing documents and taking notes when required.

% \section{Preparation}

% The study environment was prepared prior to participant arrival. 