\chapter{Introduction}
\label{ch:introduction}


Schizophrenia is a mental illness that affects millions of people globally. It involves symptoms like seeing or hearing things that are not there (hallucinations), holding strong false beliefs (delusions), confused thinking, and a lack of social interaction. These symptoms significantly complicate a person's daily life, often leading to personal and social difficulties. Beyond the direct health problems, individuals with schizophrenia frequently also face judgment and due to this also unfair treatment. Dealing with this judegment is not just about fairness, it is also a crucial part of providing good mental healthcare \cite{Shrivastava2011}.

\vspace{1em}

Empathy, the ability to understand and share the feelings of others, is vital for healthcare professionals, especially when caring for patients with conditions like schizophrenia that have a lot of stigma to them. It helps medical professionals connect with patients, improve communication, and provide care that truly focuses on the patients needs. Because of this, medical education plays a key role in developing and strengthening empathetic understanding in future healthcare workers. Traditional teaching methods often do not provide the hands-on experience needed to truly grasp what it is like to live with such a complex mental illnesses \cite{Ruffalo2024}.

\vspace{1em}

To bridge this gap and provide new methods, new technologies like Virtual Reality (VR) have emerged as promising tools for teaching empathy. VR simulations allow users to experience situations from a first-person view, offering the experience of being in someone else's shoes. However, fully immersive VR environments can sometimes be confusing or overwhelming, potentially making users feel disconnected from their own reality. Mixed Reality (MR) offers a new and different approach. It blends virtual content into the user's actual surroundings and this content can even be interacted with. This method allows for engaging experiences while keeping a connection to the familiar physical world. This could potentially make the learning process safer emotionally and easier to understand.

\vspace{1em}

This thesis explores how MR technology can be used to help medical students increase empathy for individuals with schizophrenia. The foundation for the thesis comes from the need for a potentially more effective and impactful empathy education within medical training, especially for conditions whose symptoms might be hard to gauge. The reason for creating an MR solution is to use the abilities of this technology to build a hands-on learning environment. This environment aims to bridge the gap between traditional learning methods and the actual lived experience of the illness. The goal is to provide a new way for medical students to gain a deeper, more personal understanding of the challenges faced by patients with schizophrenia. To this end, this research will investigate the following question: \textbf{Can a short Mixed Reality simulation of schizophrenia symptoms effectively increase both affective and cognitive empathy in medical students, and influence their perception of individuals diagnosed with schizophrenia?}

Furthermore, it also aims to create an experiment, which allows observers from the outside to better understand what it might look like and how individuals with schizophrenia might behave in such situations.

\paragraph{Scope of the Thesis}

The scope of this thesis includes the design, creation, and testing of a new MR application. This application simulates specific symptoms of schizophrenia, giving medical students a first-hand experience with aspects of the illness. The study involves a group of medical students, assessing the immediate effects of the simulation on their empathy levels, in their role as experiencers or observers, using both measurable data and qualitative observations. Furthermore, the study looked at how important a structured educational approach is, including preparation before the simulation and a debriefing discussion afterward, to get the most educational benefit from such experiments. 

\vspace{1em}

The following sections of this thesis provide the theoretical background on empathy and schizophrenia, review existing immersive technologies used in medical education, describe the methods used to develop and evaluate the MR application, present the findings from the user study, and discuss what these results mean. Finally, the conclusions drawn from this research are presented, along with suggestions for future work in this area.

\bigskip
\noindent
\textit{For rewriting purposes, the use of generative AI has been incorporated.}