\chapter{Introduction}
\label{ch:introduction}


\section{Project Background}

The importance of empathy in medical education has received growing attention in recent years. Healthcare workers are more and more expected not only to diagnose and treat but also to understand and emotionally connect with their patients. This is important in the context of mental health especially, where symptoms are often invisible and misunderstood. Among the most stigmatized conditions is schizophrenia - a disorder that can involve auditory and visual hallucinations and delusions.

Traditional educational approaches often rely on textbooks or clinical observations, which do not fully cover the full experience of the condition. This creates a risk that future healthcare providers may view patients with schizophrenia through a lens of detachment or fear, created by this stigma, rather than compassion and understanding.

To address this gap, immersive technologies such as Virtual Reality (VR) and Mixed Reality (MR) have emerged as promising tools, for education. These technologies can simulate symptoms in a way that allows users to step into the perspective of someone with a psychiatric condition. While VR has been more widely explored in this space, MR offers unique advantages — which will be explored in this thesis.

.... 

\section{Scope of the Project}

This thesis investigates broadly speaking whether a short MR simulation of symptoms of schizophrenia can improve empathy in medical students. The project focuses on simulating common symptoms—such as voices, visual hallucinations, and perceptual disturbances—within a controlled environment.

The simulation is embedded in an educational framework that includes the simulation itself, and a structured debriefing session, while the empathy is assessed through a questionnaire. The aim is not to create a diagnostic tool or a long-term therapy intervention, but rather to develop and test a compact, repeatable experience that can serve as a valuable supplement to traditional learning methods in their studies.

The study evaluates both cognitive and affective aspects of empathy using validated instruments, such as the Jefferson Scale of Empathy and the Brief Positive and Negative Affect Schedule. It also compares the effects on students who directly experience the simulation (headset users) and those who observe it. In doing so, this project contributes to the growing body of research on immersive empathy training, offering insights into the educational potential of MR technologies in medical settings.


....
