\chapter{Introduction}
\label{ch:introduction}


Schizophrenia is a mental illness that affects millions of people globally. It involves symptoms like seeing or hearing things that are not there (hallucinations), holding strong false beliefs (delusions), confused thinking, and a lack of social interaction. These symptoms significantly complicate a person's daily life, often leading to personal and social difficulties. Beyond the direct health problems, individuals with schizophrenia frequently also face judgment and due to this also unfair treatment. Dealing with this judegment is not just about fairness, it is also a crucial part of providing good mental healthcare.

\vspace{1em}

Empathy, the ability to understand and share the feelings of others, is vital for healthcare professionals, especially when caring for patients with conditions like schizophrenia that have a lot of stigma to them. It helps medical professionals connect with patients, improve communication, and provide care that truly focuses on the patient's needs. Because of this, medical education plays a key role in developing and strengthening empathetic understanding in future healthcare workers. Traditional teaching methods often do not provide the hands-on experience needed to truly grasp what it is like to live with such a complex mental illnesses.

\vspace{1em}

Recognizing these limitations, new technologies like Virtual Reality (VR) have emerged as promising tools for teaching empathy. VR simulations allow users to experience imagined situations from a first-person view, offering a the experience of being in someone else's shoes. However, fully immersive VR environments can sometimes be confusing or overwhelming, potentially making users feel disconnected from their own reality. Mixed Reality (MR) offers a new and different approach. It blends virtual content into the user's actual surroundings and this content can even be interacted with. This method allows for engaging experiences while keeping a connection to the familiar physical world. This could potentially make the learning process safer emotionally and easier to understand.

\vspace{1em}

This thesis explores how Mixed Reality technology can be used to help medical students develop empathy for individuals with schizophrenia. The projects foundation comes from the need for more effective and impactful empathy education within medical training, especially for conditions that are often misunderstood and whose symptoms might be hard to gauge. The reason for creating an MR-based solution was to use the abilities of this technology to build a hands-on learning environment. This environment aims to bridge the gap between textbook knowledge and the actual lived experience of the illness. The goal was to provide a new way for medical students to gain a deeper, more personal understanding of the challenges faced by patients with schizophrenia. Furthermore, the goal is also to increase the engagement of participants during the simulation, allowing those watching the person having the experience, the observers, to better understand what it might look like and how individuals with schizophrenia might behave in such situations.

\section{Scope of the Project}

The scope of this project includes the design, creation, and testing of a new Mixed Reality application. This application simulates specific symptoms of schizophrenia, giving medical students a first-hand experience with aspects of the illness. The study involved a group of medical students, assessing the immediate effects of the simulation on their empathy levels, using both measurable data and qualitative observations. Furthermore, the project looked at how important a structured educational approach is, including preparation before the simulation and a debriefing discussion afterward, to get the most educational benefit from such immersive experiences. 

\vspace{1em}

The following sections of this thesis provide the theoretical background on empathy and schizophrenia, review existing immersive technologies used in medical education, describe the methods used to develop and evaluate the MR application, present the findings from the user study, and discuss what these results mean. Finally, the conclusions drawn from this research are presented, along with suggestions for future work in this area.

\bigskip
\noindent
\textit{For rewriting purposes, the use of generative AI has been incorporated.}