\chapter{Discussion}
\label{ch:discussion}

\emph{In this section, you should discuss your result and your work. Summarize and discuss your results,  discuss your initial choices and compare with other works from the state of the art. How do you compare (if you can) ? Discuss your research questions in the light of your results. }



\subsection{Interpretation of Changes}

commentary on findings, potential explanations for lack or presence of effect, consideration of variability among participants, comment on potential improvements


While no significant change was observed in the overall empathy score, subtle shifts were detected in emotion-specific responses. For example, participants reported feeling slightly more “distressed” and less “ashamed” when thinking about individuals with schizophrenia post-intervention. However, these changes were not statistically robust in this sample.

Several possible explanations exist:
\begin{itemize}
  \item A \textbf{ceiling effect} may have limited sensitivity, as pre-evaluation empathy scores were already high.
  \item The \textbf{short duration} between evaluations may not have allowed deeper attitude changes to form.
  \item Emotional responses may be more \textbf{situational or reactive} and less stable than overall empathy attitudes.
\end{itemize}

Further qualitative feedback could provide richer insight into participant experiences.