\chapter{Discussion}
\label{ch:discussion}

\emph{In this section, you should discuss your result and your work. Summarize and discuss your results,  discuss your initial choices and compare with other works from the state of the art. How do you compare (if you can) ? Discuss your research questions in the light of your results. }

This project set out to explore whether a short MR simulation could help increase both cognitive and emotional empathy among medical students by immersing them in the experience of schizophrenia symptoms. After designing, implementing, and evaluating the simulation through a carefully planned user study, several key observations emerged that are worth reflecting on — both in terms of what worked well and what could be improved in future versions.

\section{Interpretation of Changes}

commentary on findings, potential explanations for lack or presence of effect, consideration of variability among participants, comment on potential improvements


While no significant change was observed in the overall empathy score, subtle shifts were detected in emotion-specific responses. For example, participants reported feeling slightly more “distressed” and less “ashamed” when thinking about individuals with schizophrenia post-intervention. However, these changes were not statistically robust in this sample.

Several possible explanations exist:
\begin{itemize}
  \item A \textbf{ceiling effect} may have limited sensitivity, as pre-evaluation empathy scores were already high.
  \item The \textbf{short duration} between evaluations may not have allowed deeper attitude changes to form.
  \item Emotional responses may be more \textbf{situational or reactive} and less stable than overall empathy attitudes.
\end{itemize}

Further qualitative feedback could provide richer insight into participant experiences.


they did not look up for their task, which we should've forced a bit more,
we don't know for sure if we are portraying the experience in a truthful way,
they reported that they felt ashamed to move their hands to be judged by their peers,
they were not able to focus on the task, which is a good thing, but we should've forced them to look up more often,
the stains were actually useless
small sample size and lack of statistical power may have masked potential effects,

\subsection{Quantitative and Qualitative Outcomes}

The pre- and post-simulation scores showed trends that support the initial hypothesis — that MR simulations can help increase both cognitive and affective empathy. While the sample size was relatively small, the combination of quantitative measures (like the Jefferson Scale of Empathy and B-PANAS) with qualitative feedback from debriefing sessions offered a well-rounded picture. Participants described the experience as eye-opening and emotionally engaging. Some said it changed the way they would approach patients with schizophrenia in the future — a key educational outcome.

These responses reflect similar results found in earlier empathy studies using VR tools, such as those by Hsia et al. and Formosa et al., but they also highlight how MR may provide a more grounded and ethically sound alternative to full immersion \cite{Hsia2022, Formosa2018}.

\subsection{Emotional Impact Without Harm}

Participants reported a strong emotional reaction to the simulation — not because it was frightening or traumatic, but because it felt confusing, intrusive, and unsettling. These were exactly the emotions the simulation aimed to evoke, reflecting how people with schizophrenia often describe their experience of hallucinations. Importantly, the simulations brief duration and real-world context helped to keep this emotional intensity within a safe and manageable range.

As noted in the literature, simulations like this work best when they are emotionally impactful but ethically grounded. The decision to include debriefing steps after the simulation helped frame the experience in a responsible way, encouraging participants to reflect rather than simply react \cite{Rueda2020, Ando2011}.

\subsection{Empathy Through Experience and Observation}

Another interesting finding is the role of observers — students who watched the simulation without wearing the headset. While most empathy training tools focus on the first-person experience, this study showed that observers also reported emotional and perceptual changes. Observing a peer struggle with hallucinations seemed to create a sense of shared vulnerability and empathy. This aligns with research suggesting that empathy can also grow through observation, not just direct experience \cite{Formosa2018}.

This has important implications: in educational settings where not every student can use a headset, it may still be valuable to have them witness the experience. Combining first-person and observer-based empathy learning could be a promising direction for future training tools.


\subsection{Comparison with State of the Art}
The findings align with some existing literature on empathy training, particularly in the context of schizophrenia simulations. While previous studies have shown that immersive experiences can enhance empathy, the results here suggest that the specific design and implementation of the simulation play a crucial role in determining outcomes.
For instance, while some VR-based studies reported significant increases in empathy scores \cite{Martingano2021, Ventura2020}, the current MR approach did not yield the same level of change. This may be due to differences in simulation design, duration, or participant engagement levels.
Additionally, the lack of significant change in cognitive empathy contrasts with findings from other studies that have successfully enhanced both affective and cognitive empathy through immersive experiences \cite{Rueda2020, Ando2011}. This discrepancy highlights the need for further exploration into how different modalities (MR vs. VR) and content delivery methods impact empathy outcomes.


\section{Limitations and Lessons}

Despite its strengths, the study had some limitations. The most obvious is the rather small participant group (29 students), which makes it difficult to generalize the results. Additionally, the French version of the Jefferson Scale of Empathy used in the study was translated by the author and not formally validated, which could affect reliability.

Moreover, while MR provides a strong sense of realism, it still cannot fully simulate the unpredictable, chaotic nature of schizophrenia symptoms. Some participants mentioned that they felt like they were acting or playing a role, rather than truly experiencing psychosis. This points to a challenge faced by all simulation-based education: no matter how immersive, it remains a simulation.

Finally, while observers appeared to benefit emotionally, it is still unclear how their experiences compare in depth and quality to those who wore the headset. Future studies could explore this further with larger sample sizes and long-term follow-ups.

\section{Future Opportunities}

This project only scratches the surface of what MR can offer in the field of empathy training. There is also potential to expand the simulation design: adding real storylines or voice personalization could make the experience even more powerful. And finally, pairing simulations like this with classroom teaching and patient interaction could create a holistic training model that enriches the experience.
