\chapter{Discussion}
\label{ch:discussion}

This project set out to explore whether a short Mixed Reality (MR) simulation could help increase both cognitive and affective empathy among medical students by immersing them in the experience of schizophrenia symptoms. After designing, implementing, and evaluating the simulation through a carefully planned user study, several key observations emerged that are worth reflecting on, both in terms of what worked well and what could be improved in future versions.

\section{Interpretation of Findings}

While no significant change can be observed in the overall empathy score, subtle shifts were detected in emotion-specific responses. For example, participants reported feeling slightly more “distressed” and less “ashamed” when thinking about individuals with schizophrenia post-intervention. However, none of these changes are not statistically robust in this sample.

Several possible explanations exist for the lack of significant change in empathy scores:
\begin{itemize}
  \item Pre-evaluation empathy \textbf{scores were already high}, so there may have been limited room for improvement.
  \item The \textbf{short duration} between evaluations may not have allowed deeper attitude changes to form.
  \item Emotional responses may be more \textbf{situational or reactive} and less stable than overall empathy attitudes.
  \item Individuals interaction with the simulation was \textbf{variable}, but for the most part they did not interact a lot with the simulation, which may have limited the potential for empathy growth.
  \item Interesting thoughts or reflections were only shared during the \textbf{debriefing session}, which have not been captured in the quantitative measures.
\end{itemize}

Further qualitative feedback could provide richer insight into participant experiences.


\paragraph{Qualitative outcomes}
While the sample size was relatively small, the combination of quantitative measures (JSE and B-PANAS) with qualitative feedback from debriefing sessions offered a well-rounded picture. Participants described the experience as eye-opening and emotionally engaging. Some said it changed the way they would approach patients with schizophrenia in the future — a key educational outcome. These responses reflect similar results found in earlier empathy studies using VR tools, such as those by Hsia et al. and Formosa et al., but they also highlight how MR may provide a more grounded and ethically sound alternative to full immersion \cite{Hsia2022, Formosa2018}.

\paragraph{Emotional Impact}

Participants reported a strong emotional reaction to the simulation — not because it was frightening or traumatic, but because it felt confusing, intrusive, and unsettling. These are exactly the emotions the simulation aimed to evoke, reflecting how people with schizophrenia often describe their experience of hallucinations. Importantly, the simulations brief duration and real-world context helped to keep this emotional intensity within a safe and manageable range.

\vspace{1em}

As noted in the literature, simulations like this work best when they are emotionally impactful but one needs to make sure to keep the user still in their 'real' enviornment. The decision to include debriefing steps after the simulation helped frame the experience in a responsible way, encouraging participants to reflect rather than simply react \cite{Rueda2020, Ando2011}.

\paragraph{Observation}

Another interesting finding is the role of observers — students who watched the simulation without wearing the headset. While most empathy training tools focus on the first-person experience, this study showed that observers also show potential to have  emotional and perceptual changes. Observing a peer struggle after the simulation during the debrief, seemed to create a sense of shared vulnerability and empathy. This aligns with research suggesting that empathy can also grow through observation, not just direct experience \cite{Formosa2018}.
\vspace{1em}

This has important implications: in educational settings where not every student can use a headset, it may still be valuable to have them witness the experience. Combining first-person and observer empathy learning could be a promising direction for future training tools. However, it can be assumed that the depth of emotional engagement and cognitive empathy may differ between those who directly experience the simulation and those who observe it. This could be an area for further research, particularly in how to best integrate observer roles into empathy training. 

\vspace{1em}

Furthermore, the design of the simulation itself may have influenced these outcomes. In hindsigth, the choice to assume the user will be able look around and interact with the environment was perhaps too ambitious. Many participants reported that they were ashamed to interact in front of their peers or that they tried so much to ignore the auditory hallucinations and focus on their task, which may have limited their ability to engage fully with the simulation. This suggests that future iterations could benefit from more structured tasks or auditory hallucinations to encourage deeper interaction with the environment.

\paragraph{Cognitive and Affective Empathy}
Interestingly, the lack of significant change in cognitive or affective empathy contrasts with findings from other studies that have successfully enhanced both affective and cognitive empathy through immersive experiences \cite{Rueda2020, Ando2011}. This discrepancy may be due to the specific design and implementation of the simulation, as well as the relatively short duration of the experience. On the other hand, the emotional impact of the simulation is evident in participants' qualitative feedback, suggesting that while both the cognitive and affective empathy may not have shifted statistically significant, affective responses are still there and the participants showed emotional engagement to the simulation.

\paragraph{Conclusion}
Overall, the findings align with some existing literature on empathy training, particularly in the context of schizophrenia simulations. While previous studies have shown that immersive experiences can enhance empathy, the results here suggest that the specific design and implementation of the simulation play a crucial role in determining outcomes. For instance, while some VR-based studies reported significant increases in empathy scores \cite{Martingano2021, Ventura2020}, the current MR approach did not yield the same level of change. This may be due to the reasons stated above.


\section{Limitations}

Despite its strengths, the study had some limitations. The most obvious is the rather small participant group (29 students), which makes it difficult to generalize the results. A larger sample size would provide more robust data and allow for more nuanced analysis of individual differences in empathy development.

\paragraph{Measurement Tools} Additionally, the French version of the Jefferson Scale of Empathy used in the study is translated by the author and not formally validated, which could affect reliability. While the JSE is a well-established tool, the translation process may have introduced some inconsistencies. Future studies should consider using validated translations or original language versions to ensure comparability with existing research. Additionally, the B-PANAS scale terms are chosen to reflect the immediate change in the students emotional states. Those could be changed to different terms also present in the B-PANAS but are not present in this study. As the results showed no, or only slight changes between the pre- and post-evaluation it might make sense to adapt the scale. 

\vspace{1em}

Furthermore, the anonymous nature of the questionnaires meant that it is impossible to link individual pre- and post-simulation responses. Although the same participants completed both questionnaires, the absence of a mapping mechanism prevented a direct comparison of individual empathy changes. This limited the ability to track the personalized impact of the simulation and analyze individual trajectories of empathy development.


\paragraph{Simulation Realsim} Moreover, while MR provides a strong sense of realism, it still cannot fully simulate the unpredictable, chaotic nature of schizophrenia symptoms. This points to a challenge faced by all simulation-based education: no matter how immersive, it remains a simulation. 

\paragraph{Timing of Evaluations} Another issue could have been the short time between pre- and post-evaluations. While this was intended to ensure participants are not biased by each other and could evaluate strictly the simulation, it may not have allowed enough time for deeper cognitive or emotional changes to take effect. Empathy might require more time to manifest measurable shifts after an immersive experience. Longer-term follow-up evaluations could provide more insight into the lasting impact of the simulation on empathy and whether initial changes are sustained.

\paragraph{Subjectivity of Self-Reported Empathy} It is also important to consider the inherent subjectivity of self-reported empathy. Participants were asked to evaluate their own empathy, which can be influenced by various factors such as self-perception, social desirability bias, and their understanding of empathy itself. Research suggests that individuals may evaluate their empathy differently, and self-assessment might not always align with behavioral measures of empathy \cite{Sunahara2022}.

\paragraph{Technology Used} Finally, the technology used in the simulation, while functional, is not without its limitations. Wearing the MR headset means that there is a visible device present, which may also serve as a barrier between the person experiencing the simulation and the observers. Observers are not able to see the experiencers eyes, which may limit the understanding of the emotional state of the person wearing the headset. Furthermore, in some groups the audio was leaking through the headphones we used. This means that the experience was not solely for the experiencer, but observers could also hear the audio which may have influenced their responses and did not create the intended experimental conditions.

\vspace{1em}

Finally, while some observers appeared to benefit emotionally after the debrief, it is still unclear how their experiences compare in depth and quality to those who wore the headset. 

\section{Future Directions}

This project raised many valuable questions and ideas for how MR simulations could be improved and explored further in the future. Although the simulation showed promise, there are several ways to build on these findings and develop the experience more effectively for future educational use.

\paragraph{Simulation Engagement} First, the simulation design could be improved to encourage more interaction. Many participants reported that they felt hesitant or even embarrassed to move around or interact with the virtual environment, especially while others were watching. This limited how deeply they could engage with the experience. In future versions, it may help to include clearer guidance or more structured tasks that require the user to look around, respond to specific elements, or make choices during the simulation. This could help draw users more fully into the scenario and allow for a stronger emotional and cognitive response.

\vspace{1em}

In addition to that, it would be useful to align the task instructions during the experiment to the simulation content. As the task included a lot of looking down on a piece of paper, it did not trigger or encourge the user to look around properly and some visual hallucinations which are placed in the surroundings of the participant were missed. This could be improved by either changing the task to include more interaction with the environment or by adding more visual elements that are directly related to the task, so that the user is encouraged even more to look around and engage with the virtual world.

\paragraph{Data Collection} Second, future studies should consider running the simulation with larger groups of students. A bigger sample size would make it easier to see whether small changes are meaningful or just due to chance. It could also be useful to gather follow-up data some time after the simulation, days or weeks later. This would help to understand whether the effects of the simulation last beyond the initial emotional impact. Empathy is a complex skill that may take time to change, so longer-term evaluation would give a clearer picture of how the simulation influences students attitudes and behaviors.

\vspace{1em}

Additionally, we learned a valuable lesson about how we collected our data. In the future, we need to be more careful with how we map questionnaires. Even though our questionnaires were anonymous, we could not link a participant's "before" answers to their "after" answers. To fix this while keeping things private, we could ask participants to create a simple, anonymous nickname or code at the start. That way, their answers remain completely confidential, but we can still connect their pre- and post-experiment responses to see how their individual empathy scores might have changed. This would provide much richer and more reliable data.

\paragraph{Qualitative Insights} In addition, future versions could include more qualitative methods, such as open-ended interviews or written reflections. These would allow participants to express their reactions in more detail, giving insight into how the experience affected them personally. This would also help explain why some people seemed strongly affected while others were not.

\paragraph{Observer Experience} Finally, future studies should pay close attention to the experience of observers. While most empathy simulations focus on first-person immersion, this study showed that simply watching someone else go through the experience can also have an emotional impact and potentially grow the understanding of how a person with such symptom could behave. Exploring ways to involve observers more actively, potentially through guided reflection or group discussion, could make the simulation more effective for everyone in the room, not just the person wearing the headset.

\vspace{1em}

Overall, the results of this project point to the potential of MR tools for empathy education. With a few design improvements, more structured tasks, and thoughtful follow-up methods, this approach could be a powerful addition to medical and psychological training programs.