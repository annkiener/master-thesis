\chapter{Background}
\label{ch:background}

Short abstract to the chapter (a quick summary of all the sections)

\section{Immersive Schizophrenia Simulations}
Schizophrenia is a complex mental disorder, characterized by symptoms such as auditory and visual hallucinations \cite{Silverstein2021}. In recent years, immersive technologies such as Virtual Reality (VR), Augmented Reality (AR), and Mixed Reality (MR) have emerged as powerful tools to provide first-person, interactive simulations of schizophrenia symptoms. Important to note is, that out of these three methods, VR is the most popular and most researched tool \cite{Kuhail2022} indicating that there exists a research gap concerning the other two methods. These simulations aim to enhance empathy, reduce stigma, and improve clinical understanding by offering users a direct, experiential perspective. This section explores the background of immersive schizophrenia simulations and their impact on medical education and public awareness.

% talk about the studies and the methods --> what did they do, what did they find, what are the limitations, what are the benefits, what are the gaps in the research

A lot of the research in this area has focused on the use of VR to simulate schizophrenia symptoms. For example, \cite{Kuhail2022} conducted a study where they used VR to simulate auditory hallucinations in medical students. They found that the experience increased empathy and understanding of the condition. Similarly, \cite{Silverstein2021} used VR to simulate visual hallucinations in a group of laypeople and found that it reduced stigma towards people with schizophrenia. These studies suggest that immersive simulations can be an effective tool for educating people about schizophrenia and reducing stigma. However, there is still much to learn about the potential benefits and limitations of these simulations and how they can be used most effectively.

\section{Empathy}

% talk about the studies which examine the stigma and explain the difference to empathy
% talk about the methods to evaluate empathy in different studies
% talk about the benefits of increased empathy and how it can be used in medical education
% talk about the limitations of the studies and the gaps in the research


