% Include: consent form, testing day protocol, OK from CEP committee (?), link to github for code, link to simulation video, questionnaires

\chapter{JSE Items}
\section*{Appendix A: JSE Items and Classification}
\label{app:jse}

\begin{table}[H]
\centering
\begin{tabular}{p{11cm}cc}
\toprule
\textbf{JSE Item} & \textbf{Cognitive} & \textbf{Affective} \\
\midrule
1. My understanding of how my patients and their families feel does not influence medical or surgical treatment. &  & X \\
2. My patients feel better when I understand their feelings. &  & X \\
3. It is difficult for me to view things from my patients’ perspectives. & X & \\
4. I consider understanding my patients’ body language as important as verbal communication in caregiver-patient relationships. & X & \\
5. I have a good sense of humor that I think contributes to a better clinical outcome. & Ambiguous & Ambiguous \\
6. Because people are different, it is difficult for me to see things from my patients’ perspectives. & X & \\
7. I try not to pay attention to my patients’ emotions in history taking or in asking about their physical health. &  & X \\
8. Attentiveness to my patients’ personal experience does not influence treatment outcomes. &  & X \\
9. I try to imagine myself in my patients’ shoes when providing care to them. & X & \\
10. My patients value my understanding of their feelings which is therapeutic in its own right. &  & X \\
11. Patients’ illnesses can be cured only by medical or surgical treatment; therefore, emotional ties to my patients do not have a significant influence on medical or surgical outcomes. & (X) & X \\
12. Asking patients about what is happening in their personal lives is unhelpful in understanding their physical complaints. & (X) & X \\
13. I try to understand what is going on in my patients’ minds by paying attention to their non-verbal cues and body language. & X & \\
14. I believe that emotion has no place in the treatment of medical illness. &  & X \\
15. Empathy is a therapeutic skill without which success in treatment is limited. &  & X \\
16. An important component of the relationship with my patients is my understanding of their emotional status, as well as that of their families. &  & X \\
17. I try to think like my patients in order to render better care. & X & \\
18. I do not allow myself to be influenced by strong personal bonds between my patients and their family members. &  & X \\
19. I do not enjoy reading non-medical literature or the arts. & Ambiguous & Ambiguous \\
20. I believe that empathy is an important therapeutic factor in medical or surgical treatment. &  & X \\
\bottomrule
\end{tabular}
\caption{Classification of JSE Items by Empathy Dimension (Cognitive vs. Affective)}
\label{tab:jse_classification}
\end{table}


\chapter{Shortened JSE Item Set}
\label{app:jse-short}

\section*{Appendix B: Reduced Set of JSE Items and Classification}

\begin{table}[H]
    \centering
    \begin{tabular}{p{10.5cm}cc}
    \toprule
    \textbf{Selected JSE Item} & \textbf{Cognitive} & \textbf{Affective} \\
    \midrule
    2. My patients feel better when I understand their feelings. & & X \\
    3. It is difficult for me to view things from my patients’ perspectives. & X & \\
    6. Because people are different, it is difficult for me to see things from my patients’ perspectives. & X & \\
    7. I try not to pay attention to my patients’ emotions in history taking or in asking about their physical health. & & X \\
    9. I try to imagine myself in my patients’ shoes when providing care to them. & X & \\
    10. My patients value my understanding of their feelings which is therapeutic in its own right. & & X \\
    12. Asking patients about what is happening in their personal lives is unhelpful in understanding their physical complaints. & & X \\
    13. I try to understand what is going on in my patients’ minds by paying attention to their non-verbal cues and body language. & X & \\
    14. I believe that emotion has no place in the treatment of medical illness. & & X \\
    15. Empathy is a therapeutic skill without which success in treatment is limited. & & X \\
    16. An important component of the relationship with my patients is my understanding of their emotional status, as well as that of their families. & & X \\
    17. I try to think like my patients in order to render better care. & X & \\
    20. I believe that empathy is an important therapeutic factor in medical or surgical treatment. & & X \\
    \bottomrule
    \end{tabular}
    \caption{Reduced JSE item set used in this study with classification into empathy components}
    \label{tab:jse_shortened}
    \end{table}