\chapter{Methodology}
\label{ch:method}

In this chapter, I present the methodology used to design, implement, and evaluate a MR simulation aimed at increasing empathy toward individuals diagnosed with schizophrenia. Building on the gaps and opportunities identified in the state of the art (Chapter~\ref{ch:background}), this study explores whether a brief MR simulation of symptoms—lasting approximately 3 to 4 minutes—can significantly influence both \textit{affective} and \textit{cognitive} empathy among medical students. The approach combines immersive technology, tested on medical students which already have experience with patients and know about schizophrenia and its symptoms, and guided debrief to examine how such a simulation may reshape students' perceptions and attitudes toward people with schizophrenia.

\section{Research Question}

The central research question of this project is:

\begin{quote}
\textit{\textbf{Can a short Mixed Reality simulation of schizophrenia symptoms effectively increase both affective and cognitive empathy in medical students, and influence their perception of individuals diagnosed with schizophrenia?}}
\end{quote}

This question emerges from several key insights presented in the state of the art:

\begin{itemize}
    \item VR has been shown to enhance affective empathy, but its effects on cognitive empathy are inconsistent \cite{Martingano2021, Ventura2020}.
    \item MR remains underexplored, yet early studies suggest it can balance immersion and realism, potentially supporting more empathetic outcomes \cite{Silva2017, Krogmeier2024}.
    \item Ethical concerns require immersive experiences to be framed through knowledge delivered beforehand and reflection/debriefing after the simulation to avoid stigma or stereotype reinforcement \cite{Rueda2020, Ando2011}.
\end{itemize}

\section{Using Mixed Reality}

Based on the literature review, MR offers several advantages over VR in the context of schizophrenia simulations, making it a particularly suitable choice for this thesis. One of the most important benefits of MR is its ability to provide emotional safety through the feeling of being in the real world. Unlike fully immersive VR, which can sometimes overwhelm users with intense sensory input, MR allows participants to remain grounded in their actual environment. This helps reduce the risk of distress that has been reported in VR-based schizophrenia simulations, especially when simulating frightening symptoms \cite{Zare-Bidaki2022}.

MR also offers higher relatability and engagement by integrating hallucinations and delusional content into familiar, everyday settings, such as a classroom. This contextualization can enhance the emotional resonance of the experience, as users are more likely to connect with scenarios that resemble their own daily environments \cite{Krogmeier2024}. Rather than experiencing psychotic symptoms in abstract or exaggerated virtual spaces, participants see these symptoms unfold in realistic and meaningful contexts, increasing the perceived authenticity of the simulation.

\vspace{1em}

Furthermore, MR supports a more balanced approach to empathy training by addressing both cognitive and affective components. While VR often elicits strong emotional reactions, MR allows users to emotionally engage with the simulation while still having the cognitive space to process and reflect on what they are experiencing. This engagement is particularly valuable in educational settings, where the goal is not only to generate emotional impact but also to foster a deeper understanding of the condition which is being simulated \cite{Martingano2021, Rueda2020}.

From a technical perspective, MR provides flexibility through the use of modern headsets equipped with passthrough functionality, such as the Meta Quest 3\footnote{Meta Quest 3 is a standalone mixed reality headset developed by Meta Platforms, released in October 2023. For more information: \url{https://www.meta.com/quest/quest-3/}}.
This device enables the user to see their environment, onto which simulated symptoms can be layered in real time. This technology enables the development of dynamic and responsive simulations that feel both immersive and real.


\subsection{Simulation Content Design}

The content of the MR simulation was designed to reflect commonly reported auditory and visual hallucinations described by individuals diagnosed with schizophrenia. These include whispered voices, critical comments, visual distortions, and environmental anomalies such as dark stains or disappearing objects. The structure was carefully sequenced to build in intensity over a 3–4 minute duration, beginning with subtle perceptual changes and culminating in more overt, unsettling symptoms.

The voice content was adapted from clinical interviews and voice-hearing simulations (e.g., the Hearing Voices Curriculum), grouped into twelve distinct voice groups to represent different thematic elements (e.g., confusion, fear, paranoia). Visual components included gradual screen darkening, the appearance and disappearance of dots, and symbolic imagery. These were embedded contextually within a familiar environment (e.g., a classroom) to increase emotional realism and relatability.

The simulation was prototyped using Unity and deployed on the Meta Quest 3 headset. Voice timing and synchronization with visual events were programmed using timeline-based logic, ensuring consistency across participants while maintaining immersion.

\section{Simulation Development Process}

The simulation was developed iteratively using Unity and tested internally before deployment. Initial design prototypes were reviewed by faculty members familiar with psychosis and MR development. A pilot version of the simulation was tested on a small group of peers (not part of the study population) to evaluate technical functionality, timing, and content clarity.

Feedback from these early tests led to minor revisions in voice clarity, pacing of visual hallucinations, and interaction effects. Emphasis was placed on maintaining a balance between emotional engagement and psychological safety. The final version of the simulation was finalized in collaboration with clinical educators to ensure that the depicted symptoms were both respectful and pedagogically meaningful.


\section{User Study Design and Implementation}

This methodology is directly informed by insights and gaps highlighted in the state of the art:

\begin{itemize}
    \item MR is used instead of VR to reduce overloading the senses while preserving immersion \cite{Krogmeier2024}.
    \item Debriefing sessions are included to increase meaningful reflection and avoid stigma \cite{Rueda2020, Ando2011}.
    \item The simulation content draws on real patient narratives to ensure authenticity and relatability \cite{Zare-Bidaki2022}.
    \item A short simulation duration enhances feasibility and safety without compromising emotional impact \cite{Formosa2018}.
\end{itemize}

In summary, this study adopts a structured and ethically responsible MR-based approach to schizophrenia education. The goal is to increase both \textit{affective} and \textit{cognitive} empathy in medical students by situating simulated symptoms in real-world contexts, framed by education and post-reflection. This approach addresses gaps in current VR-centric literature and showcases a rather new method for developing empathy capacity in healthcare students.


\subsection{Simulation Design Strategy}
To address the challenges mentioned above, this thesis adopts a design strategy that:

\begin{itemize}
    \item Uses Mixed Reality to simulate schizophrenia symptoms in familiar environments, allowing users to remain grounded in reality
    \item Tests the simulation on medical students which already have had a preparatory educational session to provide context and understanding of schizophrenia, reducing the risk of reinforcing stigma
    \item Includes a debriefing session to help with reflection, discussion, and ethical understanding of the experience
    \item Measures perceived immersion and empathy outcomes to evaluate the impact of the simulation on students' attitudes and understanding
    \item Uses a combination of auditory and visual hallucinations to create a layered experience that reflects the complexity of real-life symptoms
    \item Uses a gradual increase in emotional intensity, allowing users to acclimate to the experience without overwhelming them
    \item Engages students in a reflective process that encourages them to connect their experiences to real-life clinical practice and patient interactions
\end{itemize}

By doing so, this approach aims to increase both affective and cognitive empathy in medical students — helping them not only to feel what patients go through, but also to understand their experiences within a respectful and informed framework.

\subsection{Planned Evaluation and Validation}

To assess the impact of the simulation, the study uses a pretest–posttest design with validated scales measuring affective and cognitive empathy, as well as emotional perception. The primary outcomes are changes in JSE and B-PANAS scores before and after the simulation. These will be analyzed using paired-sample t-tests to evaluate within-subject effects and independent t-tests to compare headset users and observers.

Qualitative feedback collected after the simulation and during the debrief will be thematically analyzed to identify perceived realism, emotional resonance, and educational value. This combination of quantitative and qualitative methods allows for a robust, mixed-methods evaluation of the simulation’s impact on empathy and perception.



To conclude this chapter, by combining known MR technology with carefully structured educational framing, observation-based group dynamics, and post-simulation reflection, this thesis seeks to explore a multi-layered approach to empathy training in medical education. The methodology builds on known challenges and recommendations from the literature, such as avoiding emotional overload, reinforcing context, and ensuring accurate, respectful depictions of schizophrenia. 
