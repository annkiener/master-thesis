\chapter{Methodology}
\label{ch:method}

In this chapter presents the methodology used to design, implement, and evaluate a Mixed Reality (MR) simulation aimed at increasing empathy toward individuals diagnosed with schizophrenia. Building on the gaps and opportunities identified in the state of the art (Chapter~\ref{ch:background}), this study explores whether a brief MR simulation of symptoms—lasting approximately 3 to 4 minutes—can significantly influence both \textit{affective} and \textit{cognitive} empathy among medical students. The approach combines immersive technology, tested on medical students which already have experience with patients and know about schizophrenia and its symptoms, and guided debrief to examine how such a simulation may reshape students' perceptions and attitudes toward people with schizophrenia.

\section{Research Question}

The central research question of this thesis is:

\begin{quote}
\textit{\textbf{Can a short Mixed Reality simulation of schizophrenia symptoms effectively increase both affective and cognitive empathy in medical students, and influence their perception of individuals diagnosed with schizophrenia?}}
\end{quote}

This question emerges from several key insights presented in the state of the art:

\begin{itemize}
    \item VR has been shown to enhance affective empathy, but its effects on cognitive empathy are inconsistent \cite{Martingano2021, Ventura2020}.
    \item MR remains underexplored, yet early studies suggest it can balance immersion and realism, potentially supporting more empathic outcomes \cite{Silva2017, Krogmeier2024}.
    \item Ethical concerns require immersive experiences to be framed through knowledge delivered beforehand and reflection/debriefing after the simulation to avoid stigma or stereotype reinforcement \cite{Rueda2020, Ando2011}.
\end{itemize}

\section{Using Mixed Reality}

Based on the literature review, MR offers several advantages over VR in the context of schizophrenia simulations, making it a particularly suitable choice for this thesis. One of the most important benefits of MR is its ability to provide emotional safety through the feeling of being in the real world. Unlike fully immersive VR, which can sometimes overwhelm users with intense sensory input and extract them from the real world, MR allows participants to remain grounded in their actual environment. This helps reduce the risk of distress that has been reported in VR-based schizophrenia simulations, especially when simulating frightening symptoms \cite{Zare-Bidaki2022}.

MR also offers higher relatability and engagement by integrating hallucinations and delusional content into familiar, everyday settings, such as a classroom. This contextualization can enhance the emotional resonance of the experience, as users are more likely to connect with scenarios that resemble their own daily environments \cite{Krogmeier2024}. Rather than experiencing psychotic symptoms in abstract or exaggerated virtual spaces, participants see these symptoms unfold in realistic and meaningful contexts, increasing the perceived authenticity of the simulation.

\vspace{1em}

Furthermore, MR supports a more balanced approach to empathy training by addressing both cognitive and affective components. While VR often elicits strong emotional reactions, MR allows users to emotionally engage with the simulation while still having the cognitive space to process and reflect on what they are experiencing. This engagement is particularly valuable in educational settings, where the goal is not only to generate emotional impact but also to foster a deeper understanding of the condition which is being simulated \cite{Martingano2021, Rueda2020}.

\vspace{1em}

From a technical perspective, MR provides flexibility through the use of modern headsets equipped with passthrough functionality, such as the Meta Quest 3\footnote{Meta Quest 3 is a standalone mixed reality headset developed by Meta Platforms, released in October 2023. For more information: \url{https://www.meta.com/quest/quest-3/}}. This device enables the user to see their environment, onto which simulated symptoms can be layered in real time. This technology enables the development of dynamic and responsive simulations that feel both immersive and real.
\vspace{1em}
However, it is important to acknowledge certain limitations. Notably, the inability to observe the headset user's eyes from the outside poses a significant challenge for inducing genuinely "weird" or "reactive" behavior. This is because direct eye contact and generally facial cues are incredibly important for realistic social interactions. Furthermore, it is also not possible to implement certain hallucinations which have been reported by patients, because of occlusion. To be able to use this technology effectively, the hallucinations need to be simple and not occlude the user's view of their environment too much. 

\section{Design}

\paragraph{Simulation Content} The content of the MR simulation was designed to reflect commonly reported auditory and visual hallucinations described by individuals diagnosed with schizophrenia. The structure was carefully sequenced to build in intensity over a 3--4 minute duration, beginning with subtle perceptual changes and and then building up in more overt, unsettling symptoms. To reflect auditory hallucinations, voice content was adapted from clinical interviews and voice-hearing simulations (e.g., the Hearing Voices Curriculum \cite{Chaffin2013}), grouped into twelve distinct sections with voices that represent different thematic elements such as confusion, fear, and paranoia.

\vspace{1em}

The visual components were inspired by data from clinical literature. Commonly reported experiences include simple hallucinations such as flashes of white light, black stains, and colored dots that intermittently fill the visual field. Patients also describe geometric patterns, including matrix-like structures, spirals, and grid formations, often perceived as overlaid onto everyday objects. More complex hallucinations involve seeing people (either familiar or unfamiliar), disembodied faces, animals such as spiders, snakes, and cows, as well as ghostly or religious figures. In some cases, hallucinations have narrative qualities, such as witnessing fire emerging from a persons mouth or hallucinated writing appearing on walls \cite{Vanommen2019,Silverstein2021}. These complex hallucinations have not been chosen for the MR simulation, as they may be too complex to simulate effectively in a short timeframe.

\vspace{1em}

The chosen visual hallucinations were namely dark stains, pulsating in the visual field of the user, and colored dots that appear and can be interacted with. Furthermore, the user will have a darker field of vision which represents a delusion, sort of a "tunnel vision". These elements are selected to reflect the common experiences of individuals with schizophrenia while remaining manageable within the short duration of the MR simulation.  These elements were integrated into a familiar simulated environment, in this project the classroom, to enhance relatability. Visual effects are synchronized with voice content using timeline-based logic to ensure consistency and immersion across participants.

\paragraph{Iterative Development Process}

The simulation was prototyped in Unity and deployed on the Meta Quest 3 headset, with audiovisual synchronization built to maintain immersion while reflecting the lived experiences of individuals with schizophrenia. Additionally, it was developed iteratively and tested internally before deployment. Initial design prototypes were reviewed by faculty members of the University of Health in Fribourg, Switzerland (Haute école de santé Fribourg, HEdS-FR) familiar with psychosis and also supervisors of this thesis familiar with MR development. A first version of the simulation was tested to evaluate technical functionality, timing, and content clarity.

\vspace{1em}

Feedback from these early tests led to minor revisions in voice clarity, pacing of visual hallucinations, and interaction effects. Emphasis was placed on maintaining a balance between emotional engagement and psychological safety. The final version of the simulation was finalized in collaboration with experts in the healthcare domain to ensure that the depicted symptoms were both respectful and pedagogically meaningful.

\paragraph{Planned Evaluation and Validation}

To assess the impact of the simulation, we propose to design an experiment including a pretest–posttest design with debriefing and validated scales measuring affective and cognitive empathy, as well as emotional perception. The primary outcomes are changes in validated empathy scores before and after the experiment. The students which are recruited for this user study are all studying at the University of Health in Fribourg, Switzerland (in French: Haute école de santé Fribourg, HEdS-FR). The experiment features one student experiencing the simulation with the headset and other students which have the observer-role, in order for us to see if there can be any change in empathy also due to observation of another person experiencing a psychosis. The user study design is explained in more detail in Chapter~\ref{ch:userstudy}.

Qualitative feedback collected after the experiment and during the debrief will be thematically analyzed to identify perceived realism, emotional resonance, and educational value. This combination of quantitative and qualitative methods allows for a robust, mixed-methods evaluation of the experiment impact on empathy and perception.


\section{Key Factors}
To address the challenges mentioned in \ref{sec:limitationsempathytraining} and \ref{sec:ethicalchallenges}, this thesis adopts a design strategy that:

\begin{itemize}
    \item Uses Mixed Reality to simulate schizophrenia symptoms in familiar environments, allowing users to remain grounded in reality
    \item Tests the simulation on medical students which already have had a preparatory educational session to provide context and understanding of schizophrenia, reducing the risk of reinforcing stigma
    \item Includes a debriefing session to help with reflection, discussion, and ethical understanding of the experiment
    \item Measures perceived immersion and empathy outcomes to evaluate the impact of the experiment on students attitudes and understanding
    \item Uses a combination of auditory and visual hallucinations to create a layered experience that reflects the complexity of real-life symptoms
    \item Uses a gradual increase in emotional intensity, allowing users to acclimate to the experience without overwhelming them
    \item Engages students in a reflective process that encourages them to connect their experiences to real-life clinical practice and patient interactions
    \item A design limitation means only one student uses the headset at a time, and not being able to see their eyes from the outside makes it harder to create truly reactive social situations for observers
\end{itemize}

This study adopts a structured and ethically responsible MR-based approach to schizophrenia education. The goal is to increase both \textit{affective} and \textit{cognitive} empathy in medical students by situating simulated symptoms in real-world contexts, framed by education and post-reflection. 

\vspace{1em}

To sum up, this chapter presents a multi-layered approach to empathy training in medical education using MR technology. By combining immersive simulation with group discussions, and time for reflection afterward, the aim is to help students better understand the experiences of people with schizophrenia. The proposed approach was designed carefully, following advice from existing research to avoid overwhelming emotions, provide clear context, and portray schizophrenia in an accurate and respectful way.
