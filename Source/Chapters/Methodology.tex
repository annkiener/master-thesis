\chapter{Methodology}
\label{ch:method}

\emph{In this section, you should present what you intend to do, why you are doing it this way. Outline your research question in regards to the related work you presented. Explain the intended general architecture, design of your system or interfaces. Try to argument your choices with notions from the state of the art.}

This chapter outlines the methodological approach taken in this thesis to investigate the impact of Mixed Reality (MR) simulations on empathy toward individuals with schizophrenia. Drawing upon insights from the existing literature on immersive technologies and empathy training, this section details the research question, the rationale behind using MR, the design of the MR simulation system, and the approach to evaluating its effectiveness.

\section{Research Objective and Question}

As identified in the state of the art, immersive technologies have shown significant potential in mental health education, especially in fostering empathy and understanding toward individuals with schizophrenia. However, the majority of research has focused on Virtual Reality (VR), with relatively fewer studies exploring the potential of Mixed Reality (MR) for this purpose. This thesis seeks to fill that gap by investigating:

\begin{quote} \textbf{To what extent can a Mixed Reality (MR) simulation enhance cognitive and affective empathy toward individuals with schizophrenia among medical students?} \end{quote}

This research builds on studies such as those by Silva et al. (2017) and Krogmeier et al. (2024), which demonstrated that MR simulations can evoke empathy without overwhelming users, thanks to their grounding in real-world environments. 

\section{Rationale for Using Mixed Reality}

Based on the literature review, Mixed Reality (MR) offers several compelling advantages over Virtual Reality (VR) in the context of schizophrenia simulations, making it a particularly suitable choice for this thesis. One of the most important benefits of MR is its ability to provide emotional safety through real-world grounding. Unlike fully immersive VR, which can sometimes overwhelm users with intense sensory input, MR allows participants to remain anchored in their actual environment. This helps reduce the risk of distress that has been reported in VR-based schizophrenia simulations, especially when simulating severe or frightening symptoms \cite{Zare-Bidaki2022}.

MR also offers higher relatability and engagement by integrating hallucinations and delusional content into familiar, everyday settings, such as a pharmacy, classroom, or living room. This contextualization can enhance the emotional resonance of the experience, as users are more likely to connect with scenarios that resemble their own daily environments \cite{Krogmeier2024}. Rather than experiencing psychotic symptoms in abstract or exaggerated virtual spaces, participants see these symptoms unfold in realistic and meaningful contexts, increasing the perceived authenticity of the simulation.

Furthermore, MR supports a more balanced approach to empathy training by addressing both cognitive and affective components. While VR often elicits strong emotional reactions, MR allows users to emotionally engage with the simulation while still having the cognitive space to process and reflect on what they are experiencing. This dual engagement is particularly valuable in educational settings, where the goal is not only to generate emotional impact but also to foster a deeper understanding of the condition being simulated \cite{Martingano2021, Rueda2020}.

From a technical perspective, MR provides a high degree of flexibility through the use of modern headsets equipped with passthrough functionality, such as the Meta Quest Pro. These devices allow for high-resolution video feeds of the user’s environment, onto which simulated symptoms can be accurately layered in real time. This technology enables the development of dynamic, responsive simulations that feel both immersive and grounded in reality, offering a unique and effective tool for mental health education.

\section{Simulation Design}
\emph{move this to the implementation section (?)}
The Mixed Reality (MR) simulation system developed in this thesis was designed to recreate auditory and visual hallucinations inspired by first-person accounts of psychosis, with the aim of providing an immersive yet emotionally grounded educational experience. The system runs on a Meta Quest 3 headset equipped with passthrough functionality, which allows digital elements to be overlaid onto the user's real-world environment in real time. This creates a hybrid space in which simulated symptoms blend seamlessly into familiar settings like a classroom or home office.

At the core of the simulation is a orchestrator that delivers hallucinations sometimes even based on user interaction. The experience unfolds in a carefully structured sequence. It begins with auditory hallucinations: users hear unsettling whispers and hostile sentences, designed to evoke feelings of being watched, judged, or threatened—common themes in persecutory delusions. These voices are spatially anchored and vary in tone, emphasizing the sense of psychological intrusion.As the simulation progresses, visual distortions begin to appear. Dark and pulsating stains gradually emerge across the users visual field, disrupting their perception of the environment. The auditory hallucinations continue, now commenting on these visual hallucinations, deepening the immersive experience and creating a multi-sensory representation of psychosis.

In the final stage of the simulation, brightly colored, floating spheres appear above the users head. The voices urge the user to interact with them—repeating instructions until the user does so. Once the user complies, the spheres change color. This simulation draws on phenomenological reports from individuals with schizophrenia. \cite{Vanommen2019}

To support cognitive and ethical processing of this intense experience, the simulation is framed by two educational components. Before the simulation, users receive a short briefing that introduces schizophrenia, explains the diversity of symptoms, and clarifies the pedagogical purpose of the exercise. After the simulation, a guided debrief encourages open reflection, using structured questions to help participants process their emotions, connect the experience to clinical understanding, and critically engage with what they encountered.

The system design draws on established best practices in immersive mental health education, particularly the integration of experiential learning with reflection and context, as recommended by Rueda and Lara (2020) and Zare-Bidaki et al. (2022). This careful balance of sensory immersion, narrative progression, and educational framing aims to foster both empathy and insight while minimizing emotional risk.

\section{Participants and Procedure}
The target group for this study consists of medical students in their preclinical or early clinical training, specifically from the University of Health in Fribourg, Switzerland (Haute école de santé Fribourg, HEdS-FR). This population was selected for two primary reasons. First, students at this stage are actively developing their clinical attitudes, including their capacity for empathy toward patients. Second, previous research has shown that empathy training tends to be particularly effective during this formative period in a healthcare professionals education \cite{Hsia2022, Kuhail2022}.

Participation in the study is voluntary, and all participants are recruited through internal communication channels within the university. Before taking part, each participant receives comprehensive information about the study’s objectives, procedures, and potential risks. They are informed of their rights, including the ability to withdraw at any time, and are asked to sign a written consent form confirming their understanding and agreement.

The study begins with a pre-test phase, during which participants complete the Jefferson Scale of Empathy (JSE) \cite{Hojat2002} to assess their baseline empathy levels. In addition, they fill out a supplementary questionnaire that includes the Brief Positive and Negative Affect Schedule (B-PANAS) \cite{Boiroux2024}, which measures their current emotional state and perceptions regarding individuals with schizophrenia.

Following this, participants undergo the Mixed Reality (MR) simulation, which lasts approximately 4 to 5 minutes. During this time, they are exposed to a carefully designed sequence of auditory and visual hallucinations, situated in a familiar, everyday setting. The aim of the simulation is to offer an immersive experience of psychotic symptoms while maintaining a safe and relatable environment.

After completing the simulation, participants take part in a structured post-simulation debriefing session, moderated by faculty members. This reflection phase provides an opportunity to process emotional responses, share thoughts about the experience, and discuss how it relates to clinical practice. This guided reflection is a crucial element of the methodology, helping to contextualize the simulation and support the development of cognitive empathy.

To conclude the study, participants repeat the JSE and B-PANAS assessments. These post-tests are used to evaluate changes in empathy, emotional state, and attitudes toward individuals with schizophrenia. Participants are also invited to provide qualitative feedback on the simulation, including its perceived realism, emotional impact, and educational value. This mixed-methods approach ensures a comprehensive evaluation of the MR simulation’s effectiveness in fostering empathy and insight.

To conclude this chapter, by combining already known MR technology with carefully designed educational framing and post-simulation reflection, this thesis seeks to explore an approach to empathy training in medical education. The methodology builds on known challenges and recommendations from the literature, such as avoiding emotional overload, reinforcing context, and ensuring accurate, respectful depictions of schizophrenia. Through both quantitative and qualitative methods, this study will assess whether MR can provide a safer, more relatable, and more impactful alternative to traditional VR in empathy-focused mental health education.

