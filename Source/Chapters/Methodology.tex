\chapter{Methodology}
\label{ch:method}

In this chapter, I present the methodology used to design, implement, and evaluate a MR simulation aimed at increasing empathy toward individuals diagnosed with schizophrenia. Building on the gaps and opportunities identified in the state of the art (Chapter~\ref{ch:background}), this study explores whether a brief MR simulation of symptoms—lasting approximately 3-4 minutes—can significantly influence both \textit{affective} and \textit{cognitive} empathy among medical students. The approach combines immersive technology, tested on medical students which already have experience with patients and know about schizophrenia and its symptoms, and guided debrief to examine how such a simulation may reshape students' perceptions and attitudes toward people with schizophrenia.

\section{Research Question}

The central research question of this project is:

\begin{quote}
\textit{\textbf{Can a short Mixed Reality simulation of schizophrenia symptoms effectively increase both affective and cognitive empathy in medical students, and influence their perception of individuals diagnosed with schizophrenia?}}
\end{quote}

This question emerges from several key insights presented in the state of the art:

\begin{itemize}
    \item VR has been shown to enhance affective empathy, but its effects on cognitive empathy are inconsistent \cite{Martingano2021, Ventura2020}.
    \item MR remains underexplored, yet early studies suggest it can balance immersion and realism, potentially supporting more empathetic outcomes \cite{Silva2017, Krogmeier2024}.
    \item Ethical concerns require immersive experiences to be framed through knowledge delivered beforehand and reflection/debriefing after the simulation to avoid stigma or stereotype reinforcement \cite{Rueda2020, Ando2011}.
\end{itemize}

\section{Using Mixed Reality}

Based on the literature review, MR offers several advantages over VR in the context of schizophrenia simulations, making it a particularly suitable choice for this thesis. One of the most important benefits of MR is its ability to provide emotional safety through real-world grounding. Unlike fully immersive VR, which can sometimes overwhelm users with intense sensory input, MR allows participants to remain anchored in their actual environment. This helps reduce the risk of distress that has been reported in VR-based schizophrenia simulations, especially when simulating frightening symptoms \cite{Zare-Bidaki2022}.

MR also offers higher relatability and engagement by integrating hallucinations and delusional content into familiar, everyday settings, such as a classroom. This contextualization can enhance the emotional resonance of the experience, as users are more likely to connect with scenarios that resemble their own daily environments \cite{Krogmeier2024}. Rather than experiencing psychotic symptoms in abstract or exaggerated virtual spaces, participants see these symptoms unfold in realistic and meaningful contexts, increasing the perceived authenticity of the simulation.

Furthermore, MR supports a more balanced approach to empathy training by addressing both cognitive and affective components. While VR often elicits strong emotional reactions, MR allows users to emotionally engage with the simulation while still having the cognitive space to process and reflect on what they are experiencing. This engagement is particularly valuable in educational settings, where the goal is not only to generate emotional impact but also to foster a deeper understanding of the condition which is being simulated \cite{Martingano2021, Rueda2020}.

From a technical perspective, MR provides flexibility through the use of modern headsets equipped with passthrough functionality, such as the Meta Quest 3\footnote{Meta Quest 3 is a standalone mixed reality headset developed by Meta Platforms, released in October 2023. For more information, see: \url{https://www.meta.com/quest/quest-3/}}.
This device enables the user to see their environment, onto which simulated symptoms can be layered in real time. This technology enables the development of dynamic and responsive simulations that feel both immersive and real.

\section{Simulation Design}

The MR simulation developed in this thesis was designed to give students an emotional and realistic sense of what it might feel like to experience psychotic symptoms, while still keeping them in their real environment. Unlike VR, which fully replaces the users surroundings, MR allows digital symptoms — like hallucinations or sounds — to appear in the users actual space. % repetitive?

The simulation shows both auditory and visual symptoms, based on real descriptions from people who live with schizophrenia. Users hear critical or unsettling voices and see visual changes with the goal of distracting them. These effects are introduced step by step to reflect how symptoms often build gradually. The aim is not to scare or shock, but to help students connect with the emotional and mental confusion that someone with psychosis might feel.

Because the simulation is only 3 to 4 minutes long, it focuses on giving a short but meaningful experience. It is placed in a familiar environment, which is the classroom, so that the symptoms feel more relatable. This balance is important: the goal is to increase empathy and understanding, not to create fear or reinforce negative stereotypes. % repetitive?

To support this, the simulation is framed by two key points. At some point before the day of testing, students have already been lectured sometime in their studies on the topic of schizophrenia and also already have pracitcal experience with patients. They also will be briefed about the simulation and what it should show. Afterwards, they take part in a guided debrief, where they can reflect on how they felt, what they learned, and how it might change the way they see or interact with patients. This step is especially important, as it helps students process the experience in a thoughtful way.

The overall design is based on ideas from recent research, which shows that immersive tools work best when combined with education and reflection. Studies by Rueda and Lara (2020) and Zare-Bidaki et al. (2022) stress that simulations should be realistic and meaningful, but also ethically responsible and emotionally safe. This approach follows those recommendations closely, aiming to create a learning experience that supports both emotional connection and critical thinking \cite{Rueda2020,Zare-Bidaki2022}.

\section{Design Choices}

This methodology is directly informed by insights and gaps highlighted in the state of the art:

\begin{itemize}
    \item MR is used instead of VR to reduce overloading the senses while preserving immersion \cite{Krogmeier2024}.
    \item Debriefing sessions are included to increase meaningful reflection and avoid stigma \cite{Rueda2020, Ando2011}.
    \item The simulation content draws on real patient narratives to ensure authenticity and relatability \cite{Zare-Bidaki2022}.
    \item A short simulation duration enhances feasibility and safety without compromising emotional impact \cite{Formosa2018}.
\end{itemize}

In summary, this study adopts a structured and ethically responsible MR-based approach to schizophrenia education. The goal is to increase both \textit{affective} and \textit{cognitive} empathy in medical students by situating simulated symptoms in real-world contexts, framed by education and post-reflection. This approach addresses gaps in current VR-centric literature and showcases a rather new method for developing empathy capacity in healthcare students.


To conclude this chapter, by combining known MR technology with carefully structured educational framing, observation-based group dynamics, and post-simulation reflection, this thesis seeks to explore a multi-layered approach to empathy training in medical education. The methodology builds on known challenges and recommendations from the literature, such as avoiding emotional overload, reinforcing context, and ensuring accurate, respectful depictions of schizophrenia. 
