\chapter{State of the Art}
\label{ch:background}

\emph{In this section, you should outline the most important academical works revolving around your topic. You may use subsections to structure your work if necessary.}

This chapter reviews the current advancements in immersive technologies for schizophrenia simulations and therapy. It examines the use of Virtual Reality (VR), Augmented Reality (AR), and Mixed Reality (MR) in simulating psychotic symptoms, fostering empathy, and treating schizophrenia spectrum disorders. The chapter highlights key studies, their methodologies, and outcomes, while addressing design considerations, limitations, and future directions for integrating immersive technologies into healthcare and education.

\section{XR Technologies}
Extended Reality (XR) refers to the spectrum of immersive technologies that blend the physical and digital world. This includes Virtual Reality (VR), which fully immerses the user in a computer-generated environment; Augmented Reality (AR), which overlays digital content onto the real world; and Mixed Reality (MR), which combines both, allowing real and virtual elements to interact in real time. In the context of schizophrenia education, VR is often used to simulate intense sensory experiences, such as auditory or visual hallucinations. AR has been applied to embed simulated voices or visual cues into everyday settings, making the experience more relatable. MR, the focus of this thesis, seeks to integrate the strengths of both: allowing users to remain grounded in reality while experiencing interactive, layered symptoms, which may lead to higher engagement and emotional resonance \cite{Krogmeier2024, Silva2017, Zare-Bidaki2022}.

\section{Immersive Schizophrenia Simulations}
Schizophrenia is a complex mental disorder, characterized by symptoms such as auditory and visual hallucinations \cite{Silverstein2021}. In recent years, immersive technologies such as Virtual Reality (VR), Augmented Reality (AR), and Mixed Reality (MR) have emerged as powerful tools to provide first-person, interactive simulations of schizophrenia symptoms. Important to note is, that out of these three methods, VR is the most popular and most researched tool \cite{Kuhail2022} indicating that there exists a research gap concerning the other two methods. These simulations aim to enhance empathy, reduce stigma, and improve clinical understanding by offering users a direct, experiential perspective. This section explores the background of immersive schizophrenia simulations and their impact on medical education and public awareness.

\subsection{Simulation of Hallucinations and Delusions}
A major goal of schizophrenia simulations is to recreate symptoms such as hearing voices, visual hallucinations, or experiencing delusions. For example, Skoy et al. (2016) \cite{Skoy2016} created a simulation where users hear disturbing voices through headphones to better understand the kind of confusion and distraction that people with schizophrenia may deal with. Other studies, like those by Zare-Bidaki et al. (2022) \cite{Zare-Bidaki2022} and Chaffin et al. (2013) \cite{Chaffin2013}, used virtual reality (VR) to create multi-sensory experiences, combining sound, visuals, and interaction to simulate intense delusions and internal voices. These simulations give users a chance to experience how people with psychosis may perceive the world.

Visual symptoms have also been studied. Silverstein et al. (2021) \cite{Silverstein2021} and van Ommen et al. (2019) \cite{Vanommen2019} looked at how people with schizophrenia might see distorted images, such as unfamiliar faces, strange objects, or warped environments. These insights helped developers create more realistic simulations, showing how visual hallucinations might appear in everyday places. The goal of these simulations is to give users a first-person view of the symptoms. By doing this, they aim to help healthcare professionals, students, and the general public better understand what people with schizophrenia go through.

Some of these tools have been made specifically for training in medical and nursing education. Yoo et al. (2020) \cite{Yoo2020} and Lee et al. (2020) \cite{Lee2020} developed VR training programs using 360-degree video and actors to recreate clinical situations. These simulations include symptoms like hearing voices, false beliefs, or patients behaving aggressively or withdrawing. They were shown to be realistic and useful for learning, and were delivered through VR headsets.

\subsection{Virtual Reality in Schizophrenia Education}

Many studies in this field have focused on using VR to simulate schizophrenia symptoms. For example, Kuhail et al. (2022) \cite{Kuhail2022} used VR to simulate auditory hallucinations for medical students. The students who took part in the simulation showed more empathy and a better understanding of what it’s like to live with schizophrenia. Similarly, Silverstein et al. (2021) \cite{Silverstein2021} created a VR experience that simulated visual hallucinations. Participants who went through this simulation showed reduced stigma toward people with schizophrenia. These studies suggest that VR simulations can be helpful for educating people about mental illness and for breaking down stereotypes.

Marques et al. (2022) ran an experiment to compare how well a VR simulation and a regular 2D video improved empathy, knowledge, and attitudes about schizophrenia. Participants were divided into two groups: one group experienced the symptoms of psychosis in VR, and the other watched the same content in a video format. The study found that the VR group showed greater increases in cognitive empathy and more positive attitudes. However, there were also some challenges. For example,there was no control group with no exposure at all, making it harder to judge the full effect of the simulation. The short length of the simulation may have limited its impact, and some participants struggled to fully engage with the technology due to unfamiliarity with VR. In addition, the study reported some unexpected results in physiological responses and did not measure how immersive the experience felt \cite{Marques2022}. In this thesis, we aim to address some of these issues by using Mixed Reality (MR) instead of VR, which may feel more familiar and less overwhelming to users. We will also measure the level of immersion and provide participants with background knowledge before the simulation to improve understanding. % follow up on this

Another relevant study is by Zare-Bidaki et al. (2022), who looked at how a single VR simulation of psychosis (VRSP) affected medical students' stigma, empathy, and knowledge compared to traditional patient visits. Their results showed that VRSP gave students a more consistent understanding of symptoms, since real patient experiences can vary from case to case. Both the VR and patient visit groups showed reduced stigma and increased empathy, but the VR group had a slightly stronger effect. However, the study noted that VR should not replace direct patient contact - it should be used as a supplement. This real-world context adds human depth, reduces the risk of reinforcing stereotypes, and fosters more lasting empathy. Without this, users might focus solely on the strangeness or severity of symptoms rather than understanding the person behind them. Educational simulations should therefore be seen as tools that complement —n ot replace — personal connection, clinical observation, and comprehensive education \cite{Zare-Bidaki2022,Hsia2022}.
Importantly, while both methods increased empathy, the link between empathy and stigma reduction still needs more research \cite{Zare-Bidaki2022}.

The study also pointed out some possible downsides of using VR without additional context. If the simulation focuses too much on dramatic or frightening symptoms, it could unintentionally increase stigma rather than reduce it. That's why it's important to pair simulations with educational content that shows the full picture of what it's like to live with schizophrenia—not just the symptoms. In this thesis, this issue will be addressed by ensuring that students receive proper background knowledge on schizophrenia before the simulation, so they can better understand the experience from a cognitive perspective as well \cite{Zare-Bidaki2022}.

Finally, the study found that the design of the simulation matters. If the environment is too dramatic or scary, it might negatively affect how participants feel about people with schizophrenia. Using calm and neutral environments can help avoid this. The researchers also suggested that Augmented Reality (AR) could be even more effective than VR for building empathy because it allows users to see the real world while experiencing symptoms. This makes the experience feel more realistic and less overwhelming \cite{Zare-Bidaki2022}. These ideas also support the decision to use MR in this thesis. Since MR combines real-world elements with simulated symptoms, it has the potential to offer a more balanced and relatable experience.


\subsection{Augmented Reality and Mixed Reality in Schizophrenia Education}

An increasing number of studies are exploring the use of Augmented Reality (AR) and Mixed Reality (MR) in schizophrenia education. One early example is by Silva et al. (2017), who created a tool using both virtual reality (VR) and AR to simulate psychotic symptoms. This system, developed with input from psychiatric professionals, was designed to help users — especially medical students — better understand schizophrenia and reduce stigma. The AR tool allowed users to interact with simulated symptoms in real time, providing a safe and controlled learning environment \cite{Silva2017}.

To test the system, 21 medical students used AR glasses to experience the simulation. Afterward, they filled out questionnaires about their attitudes toward schizophrenia, how realistic they found the experience, and whether their views had changed. Students gave high ratings for the audio quality and educational value of the simulation. Many said it helped them better understand what psychotic experiences might feel like. However, some users also reported problems, such as discomfort from the equipment, difficulty focusing in the environment, and issues with synchronization \cite{Silva2017}.

The simulations impact on empathy and stigma was measured using questionnaires before and after the experience. The results showed that students felt more empathy, expressed more concern for a fictional patient, and were more willing to help. However, there was also a small increase in stigma scores, showing that the results were complex. The study suggests that while AR can help foster empathy, future designs should focus on improving comfort and exploring long-term effects. It also recommends combining simulations with brief educational sessions on schizophrenia to deepen understanding. In this thesis, a similar approach is taken: students will participate in the simulation after receiving a lecture on schizophrenia, which should help them better understand and relate to the experience. \emph{(Check: do all participants actually receive the lecture beforehand?)}

A more recent project by Krogmeier et al. (2024) involved the development of \textit{Live-It}, an AR simulation that used the passthrough function of the Meta Quest headset. This system simulated hallucinations and delusions in familiar places like living rooms or pharmacies. The design was based on real-life accounts from individuals with schizophrenia and was reviewed by neuropsychologists to ensure accuracy \cite{Krogmeier2024}.

Participants in the study  —mainly students and professionals in mental health — reported strong emotional reactions and said the simulation helped them better understand schizophrenia symptoms. One of the strengths of \textit{Live-It} was its ability to place symptoms into everyday situations, which made the experience feel more realistic and less overwhelming than fully immersive VR. For example, users heard voices that ranged from critical to supportive, reflecting the variety of hallucinations people might experience. The simulation also ended with hopeful, recovery-focused messages, which helped balance the emotional impact. Overall, the study found that \textit{Live-It} increased empathy and encouraged participants to support individuals with schizophrenia. It showed that AR can be a powerful tool in mental health education, especially when it helps bridge the gap between theory and real-life experience \cite{Krogmeier2024}.

Some programs are now combining simulations with direct interactions, such as discussions or debriefs with individuals who have lived experience of schizophrenia. For instance, Hsia et al. showed that pharmacy students who took part in a simulation and also heard from a guest speaker diagnosed with schizophrenia showed greater reductions in stigma and increases in empathy \cite{Hsia2022}. This combined approach helps address one of the main concerns with simulations — that they can unintentionally increase social distance or reinforce stereotypes if not supported by real-life context. Including authentic human interaction can make the experience more meaningful and well-rounded. 

\subsection{Challenges and Considerations}

Even though immersive simulations are becoming more popular in mental health education, especially for schizophrenia, they still come with some challenges. One concern is that, if not properly designed and explained, these simulations could unintentionally reinforce negative stereotypes or even increase stigma instead of reducing it \cite{Ando2011}. Some users have also reported feeling uncomfortable or distressed during the simulations — especially when the content includes intense symptoms like command hallucinations or feelings of paranoia \cite{Chaffin2013, Zare-Bidaki2022}.

Another issue is finding the right balance between realism and ease of use. Creating highly realistic simulations requires detailed sound and visual effects, as well as believable acting. This can be technically complex and time-consuming. In addition, designers must consider the wide variety of ways symptoms can appear in different individuals. Without this, the simulation might not be relevant — or even appropriate — for all users \cite{Zare-Bidaki2022}.

In summary, immersive technologies like VR, AR, and MR have great potential in mental health education, especially for conditions like schizophrenia. As these technologies continue to improve, they are expected to become more common in medical and nursing training. This could lead to more engaging and effective learning experiences, and ultimately, better care for patients.

\section{Empathy}

Empathy is a key part of good communication and care in healthcare. Many studies have shown that when healthcare professionals show empathy, patients are more satisfied, more likely to follow treatment plans, and often have better mental health outcomes \cite{Cunico2012, Olson1995, Ozcan2018}.

In medical and nursing education, empathy is no longer seen as just a "soft skill." It is now treated as something important that can be taught and developed. Teaching empathy helps improve the way future professionals connect with patients and provide care \cite{Cunico2012}.

Empathy is usually described as having two main parts: \textit{cognitive empathy} and \textit{affective empathy}. Cognitive empathy is the ability to understand what someone else is thinking or feeling. Affective empathy means actually feeling or emotionally connecting with what the other person is going through \cite{Ventura2020, Martingano2021}. In healthcare, both types are important. Understanding a patient's perspective (cognitive empathy) helps with communication and decision-making, while emotional connection (affective empathy) helps build trust and stronger relationships \cite{Cunico2012, Ozcan2018}. Some researchers also talk about a third type, called \textit{clinical empathy}, which is about clearly showing patients that you understand them and care about helping them \cite{Hojat2002}.

Understanding this and training empathy helps doctors and nurses better understand their patients and respond in helpful and compassionate ways \cite{Ozcan2018, Olson1995}. However, research has shown that empathy can decrease during medical training. This might be because students are under pressure, focusing more on technical knowledge, or feeling emotionally drained \cite{Mattsson2024, Ozcan2018}. This decline in empathy can lead to negative outcomes for both patients and healthcare professionals. Patients may feel misunderstood or neglected, while healthcare providers may experience burnout and job dissatisfaction \cite{Mattsson2024, Cunico2012}. Therefore, it is crucial to find effective ways to teach and maintain empathy in medical education.

The immersive tools discussed in this thesis are being tested in medical and nursing schools as a new way to teach empathy by letting students “step into the shoes” of patients \cite{Alieldin2024}. As mentioned earlier, this approach is becoming more popular in education and has shown promising results.

\subsection{Empathy Enhancement with Virtual Reality}

Virtual Reality (VR) has often been called the "ultimate empathy machine" because it can create powerful first-person experiences in fully immersive environments \cite{Milk2015}. Several studies support this idea, showing that VR can have a strong emotional effect on users.

VR is especially useful when it comes to helping people understand the experiences of stigmatized groups, such as individuals with schizophrenia \cite{Formosa2018, Marques2022, Mattsson2024}. These systems allow users to go through simulated versions of symptoms like hearing voices or feeling paranoid. By placing users in situations that reflect what it might be like to live with psychosis, these simulations aim to increase empathy and reduce negative attitudes. For example, Formosa et al. found that people who used a VR simulation of schizophrenia symptoms felt more empathy and showed less stigma afterwards compared to those who did not use the simulation \cite{Formosa2018}. A similar study by Hsia et al. showed that pharmacy students who experienced auditory hallucinations in VR also became more empathetic and less stigmatizing toward people with schizophrenia \cite{Hsia2022}.

Marques et al. also found that participants who used a VR simulation scored higher on empathy than those who only watched a standard 2D video \cite{Marques2022}. This supports the idea that "embodied perspective-taking" — actually feeling like you're in someone else’s position — can be more powerful than just imagining it, especially for people with little prior knowledge about mental illness.

Mattsson et al. studied how nursing students responded to VR simulations and found that many of them not only felt emotionally connected to the experience but also wanted to help the virtual patients \cite{Mattsson2024}. Students said they were better able to recognize emotional cues and understand the patient’s perspective compared to traditional learning methods.

In a related study, Olson showed that when nurses expressed empathy in their care, patients experienced less emotional distress \cite{Olson1995}. Together, these findings show the importance of empathy in healthcare and suggest that immersive technologies like VR can play a helpful role in training future professionals to connect with patients more deeply.


\subsection{Limitations of VR in Empathy Training}
A meta-analysis by Martingano et al. that looked at 43 studies found that VR tends to improve \textit{affective empathy} — the ability to emotionally connect with others — but has less impact on \textit{cognitive empathy}, which involves understanding another person’s perspective on a deeper level \cite{Martingano2021}. This means that while VR can make people feel emotionally involved, it may not be as good at helping them think through what others are going through unless additional tools, like reflection or discussion, are included. One reason for this might be that VR experiences take away the cognitive ability of the user to imagine themselves in someone else's shoes. The experience already gives you this perspective \cite{Martingano2021}. 

Some researchers have raised concerns about relying too much on VR for teaching empathy, especially in moral or educational settings. Rueda and Lara (2020) argue that the emotional reactions produced by VR can sometimes be shallow or short-lived. They suggest that what’s needed is \textit{reason-guided empathy} — where emotional experiences are paired with ethical reflection and a deeper understanding of context \cite{Rueda2020}. Without this, people might have strong feelings in the moment but not actually change how they think or behave in the long term \cite{Martingano2021}. \emph{emphasize on the debrief after the testing}

Other important limitations have also been noted. Martingano et al. found that more advanced or immersive VR experiences were not always better than simpler tools, such as 360° video, when it came to increasing empathy \cite{Martingano2021}. Many VR simulations also focus on dramatic storytelling and emotional intensity, which may reduce the user’s ability to make their own decisions during the experience. This can limit the development of perspective-taking and meaningful change in behavior.

Finally, there are ethical concerns. Rueda and Lara warn that if VR simulations are not carefully designed, they might unintentionally reinforce stereotypes or simplify complex experiences in harmful ways \cite{Rueda2020}. These concerns are especially important in mental health simulations, where realism and responsible storytelling are critical. In this thesis, these risks are taken seriously and addressed by combining immersive experience with structured preparation and post-simulation reflection to ensure a more thoughtful and lasting impact.

\subsection{Mixed Reality and Empathy in Medical Education}
\emph{show how everything aligns with MR and how it is the best option for this thesis}

While virtual reality (VR) has received the most attention in empathy training, mixed reality (MR) is becoming a promising alternative. MR allows users to stay partly connected to the real world while interacting with digital content. It combines the full immersion of VR with the real-world awareness of augmented reality. This mix can help reduce sensory overload and make the experience feel more familiar and less overwhelming—two common problems with fully immersive VR \cite{Zare-Bidaki2022}.

Recent research has shown that MR can be effective for building empathy, especially in mental health education. For example, Silva et al. (2017) tested an MR system that simulated symptoms of psychosis. Medical students who used it reported feeling more empathy and better understanding what it’s like to live with schizophrenia \cite{Silva2017}. The combination of real-world surroundings and simulated symptoms made the experience feel realistic but not too intense. In a similar project, Krogmeier et al. (2024) developed \textit{Live-It}, a passthrough MR experience that showed hallucinations in everyday places like a living room or pharmacy. Participants said the experience felt very emotional and relatable, partly because the simulation included familiar environments and ended with messages about hope and recovery \cite{Krogmeier2024}.

These studies show that MR can be a valuable tool in medical education, especially for teaching empathy toward people with mental illness. In this thesis, MR is chosen not only for its technological features, but also because it offers a better learning experience. It allows students to experience what psychotic symptoms might feel like, without fully disconnecting from the real world. This balance could help create a deeper and safer understanding of mental illness by combining emotional impact with cognitive clarity.

\subsection{Measuring Empathy}
Various instruments are used to measure these dimensions of empathy, including the Jefferson Scale of Empathy (JSE), which is being widely applied in medical education \cite{Alieldin2024}. This tool allows researchers to assess changes in empathy following interventions and distinguish between shifts in emotional versus cognitive components, which is also what I want to achieve in this thesis. In the context of this thesis, the JSE will be used to measure the impact of the MR simulation on medical students' empathy levels. The JSE is a validated instrument that has been widely used in medical education research and has demonstrated reliability and validity in assessing empathy in healthcare professionals \cite{Hojat2002}. By employing the JSE, this study aims to provide a comprehensive evaluation of the effectiveness of the MR simulation in enhancing both cognitive and affective empathy among medical students. In chapter 5, we will discuss the results of the JSE and how they relate to the overall objectives of this thesis.

\begin{table}[htbp]
    \centering
    \scriptsize
    \resizebox{\textwidth}{!}{%
    \begin{tabular}{|p{2.8cm}|p{1.4cm}|p{2cm}|p{2.4cm}|p{2.6cm}|p{3.6cm}|p{0.8cm}|p{1.6cm}|}
    \hline
    \textbf{Authors (Year)} & \textbf{Tech.} & \textbf{Target Group} & \textbf{Design} & \textbf{Symptoms Simulated} & \textbf{Main Results} & \textbf{Year} & \textbf{Empathy Type} \\
    \hline
    Skoy et al.\ (2016) & VR & Pharmacy Students & Pre/Post Exp. & Auditory Hallucinations & Increased empathy & 2016 & Affective \\
    Zare-Bidaki et al.\ (2022) & VR & Medical Students & Patient Visit Comparison & Auditory + Delusions & Greater empathy, stigma reduction & 2022 & Affective \\
    Chaffin et al.\ (2013) & VR & General Public & Exploratory & Auditory + Visual Halluc. & Understanding, distress & 2013 & Affective \\
    Silverstein et al.\ (2021) & VR & General Public & Survey/VR & Visual Hallucinations & Reduced stigma & 2021 & Affective \\
    Van Ommen et al.\ (2019) & Survey & Psychosis Patients & Survey Study & Visual Hallucinations & Simulation insights & 2019 & None \\
    Yoo et al.\ (2020) & 360 VR & Nursing Students & Usability Study & Multiple & Immersive training & 2020 & Affective \\
    Lee et al.\ (2020) & 360 VR & Nursing Students & Usability Study & Multiple & Awareness, safe tool & 2020 & Affective \\
    Kuhail et al.\ (2022) & VR & Medical Students & Controlled Trial & Auditory Hallucinations & Strong engagement & 2022 & Affective \\
    Marques et al.\ (2022) & VR & General Public & Controlled Trial & Auditory + Visual + Delusions & High empathy, physio response & 2022 & Affective \\
    Silva et al.\ (2017) & AR & Medical Students & Pilot Study & Auditory + Visual & Realism, some discomfort & 2017 & Affective \\
    Krogmeier et al.\ (2024) & AR/MR & Healthcare Students & Qualitative Eval. & Auditory + Visual + Delusions & Empathy increased & 2024 & Affective \\
    Hsia et al.\ (2022) & VR + Speaker & Pharmacy Students & Mixed-Methods & Auditory + Discussion & Speaker addition effective & 2022 & Affective \\
    Cunico et al.\ (2012) & Training & Nursing Students & Longitudinal & N/A & Emotional empathy ↑ (females) & 2012 & Affective \\
    Ozcan et al.\ (2018) & Curriculum & Nursing Students & Comparative Longitudinal & N/A & Higher empathy (integrated) & 2018 & Affective \\
    Ventura et al.\ (2020) & VR & General Pop. & Meta-analysis & Various & Cognitive empathy ↑ & 2020 & Cognitive \\
    Rueda \& Lara (2020) & VR & General & Theory Review & N/S & Affective focus & 2020 & Affective \\
    Mattsson et al.\ (2024) & VR & Nursing Students & Qualitative & Pneumonia Sim & Affective empathy elicited & 2024 & Affective \\
    Alieldin et al.\ (2024) & VR & Medical Students & Mixed Methods & Social Isolation & JSE ↑, emotional engagement & 2024 & Affective \\
    Martingano et al.\ (2021) & VR & General Pop. & Meta-Analysis & Various & Affective ↑, Cognitive — & 2021 & Affective \\
    \hline
    \end{tabular}%
    }
    \caption{Summary of studies on immersive and training-based empathy interventions, including type of empathy increased.}
    \label{tab:empathy_summary}
    \end{table}
    