\chapter{State of the Art}
\label{ch:background}

This chapter provides a review of the current research on the use of immersive technologies in simulating psychotic symptoms, particularly for the purpose of increasing empathy in healthcare education. It explores how Virtual Reality (VR), Augmented Reality (AR), and Mixed Reality (MR) have been applied in educational and also clinical settings, with the focus on schizophrenia. The chapter highlights both the promise and limitations of these technologies, outlines major research gaps and presents evidence that MR is a balanced and potentially more effective tool for empathy training. It also addresses design and ethical considerations which are critical to building realistic and meaningful simulations, and introduces the reasons behind the simulation strategy adopted in this thesis.

\section{Extended Reality (XR) Technologies}
Extended Reality (XR) refers to the spectrum of immersive technologies that blend the physical and digital worlds. This includes Virtual Reality (VR), which fully immerses the user in a computer-generated environment, Augmented Reality (AR), which overlays digital content onto the real world, and Mixed Reality (MR), which combines both, enabling real and virtual elements to interact dynamically. The development and classification of these environments can be understood through Milgram and Kishino’s Reality-Virtuality (RV) Continuum, a framework that positions real and virtual environments on a continuous scale, with Mixed Reality covering the space in between. Their accompanying taxonomy further describes experiences along three dimensions: extent of world knowledge (how much the system knows about the real environment), reproduction fidelity (how accurately it replicates real-world perception), and extent of presence metaphor (how naturally users interact within the environment) \cite{Skarbez2021}. In the context of schizophrenia, VR is often used to simulate intense experiences, such as auditory or visual hallucinations, representing psychosis. AR has been applied to embed simulated voices or visual cues into everyday settings, making the experience more relatable. MR, the focus of this thesis, seeks to integrate the strengths of both: allowing users to remain grounded in reality while experiencing interactive, layered symptoms, potentially leading to higher engagement and stronger emotional responses \cite{Krogmeier2024, Silva2017, Zare-Bidaki2022}.



\section{Immersive Simulations of Schizophrenia}
Immersive simulations have emerged as an important strategy to foster a better understanding of schizophrenia symptoms, as well as to address persistent stigma surrounding the disorder. Virtual Reality (VR) and Augmented Reality (AR) technologies are especially valuable, offering experiential learning environments where participants can “step into the shoes” of individuals experiencing hallucinations, delusions, or cognitive impairments \cite{Krogmeier2024}. These approaches have proven to be effective not only in increasing empathy and knowledge but also, in many cases, in reducing stigma among participants \cite{Krogmeier2024,Holopainen2023}.

Recent developments have also emphasized the educational use of simulations targeted specifically at healthcare students and professionals, providing controlled, safe, and replicable experiences of psychotic symptoms to better prepare them for real-world clinical interactions \cite{Yoo2020,Lee2020}.

\subsection{Virtual Reality Applications}
Virtual Reality applications in the context of schizophrenia simulations typically seek to recreate sensory and cognitive disturbances through fully immersive experiences. These applications range from fully interactive environments developed with game engines like Unity to 360-degree videos played via head-mounted displays (HMDs) \cite{Yoo2020,Lee2020}.

The use of VR allows users to experience positive symptoms of schizophrenia, such as auditory hallucinations, persecutory delusions, and visual distortions, within a safe environment. Furthermore, VR interventions are now increasingly evaluated for their usability, realism, and educational effectiveness.

\subsubsection{Simulations of Schizophrenia Symptoms in Medical Training}

\emph{If not mentioned in the intro, you may add a small section on the symptoms, because they are mentioned a lot as auditory or visual hallucinations but you will have to justify you design choices. Or this can be highlighted in the table.}


A major goal of schizophrenia simulation is to recreate symptoms such as hearing voices, visual hallucinations, or experiencing delusions. For example, studies like those by Zare-Bidaki et al. and Chaffin et al., used VR to create multi-sensory experiences, combining sound, visuals, and interaction to simulate intense delusions and internal voices \cite{Zare-Bidaki2022,Chaffin2013}. Importantly, Zare-Bidaki et al.'s participants were medical students tasked with simulating the experience of a psychotic episode to enhance their clinical empathy and understanding. Silverstein et al. and van Ommen et al. looked at how people with schizophrenia might see distorted images, such as unfamiliar faces, strange objects, or unreal environments \cite{Silverstein2021,Vanommen2019}. However, it is important to clarify that their works primarily explored the phenomenology and neurobiology of visual hallucinations in clinical schizophrenia, not through VR simulations, and not directly in educational interventions for students. Nonetheless, these insights become incredibly important, and were also heavily used for the design of the simulation created in this project, as they help to create a more realistic and relatable experience for users.

Furthermore, some tools are made specifically for training in medical and nursing education. Yoo et al. and Lee et al. developed VR training programs using 360-degree video and actors to recreate clinical situations \cite{Yoo2020, Lee2020}. The primary goal was to simulate encounters with patients exhibiting psychiatric symptoms in acute hospital settings. These tools largely rely on passive observation within pre-recorded 360° videos, meaning that while users can look around and witness events unfold, direct interaction with the environment is usually limited. Thus, while they offer vivid emotional realism, they often lack deep interactivity. These simulations included symptoms like hearing voices or patients behaving aggressively and were shown to be realistic and useful for learning. Kuhail et al. (2022) and Domnick et al. developed similar VR tools for medical students, which helped increase understanding and reduce stigma \cite{Kuhail2022, Domnick2023}.

\subsubsection{Research Gaps}

\emph{Note Marine: to be discussed : in my point of you can merge section 2.2.1.1 and 2.2.1.2. I undersand the transition (second green highlighted sentence) saying there a gap based on [12] but then you have quite an important section on AR/MR.I can suggest to have a more unit VR section (with the same content)- adding the advantages and limitations of the studies detailed than can be addressed thanks to AR/MR (as already mentionned). Or to have a structure more close to the 2.2.2}

While immersive technologies have become increasingly valuable for simulating schizophrenia symptoms, existing research remains heavily focused on VR. Among the broader XR spectrum, VR is by far the most studied and widely applied method, leaving AR and MR comparatively underexplored \cite{Kuhail2022}.

For instance, a systematic review by Holopainen et al. examined 12 studies using VR-based interventions for schizophrenia, including cognitive behavioral therapy (CBT) or social skills training. These studies reported positive outcomes across a range of symptoms — such as hallucinations, paranoia, and cognitive difficulties — with minimal adverse effects. Notably, none of the reviewed interventions utilized AR or MR, further showing the gap in the literature \cite{Holopainen2023}.

Similarly, Lan et al. reviewed a large number of articles and found that, while VR continues to show promise in these medical settings, there was no evidence of AR or MR being tested in medical trials for psychosis. Despite the many advantages these technologies could offer — particularly MR, which allows for immersive symptom simulation while keeping users aware that they remain in the real world \cite{Lan2023}.

This gap presents a good opportunity to explore MR as an alternative approach, especially for applications with the focus on empathy development. MR has the potential to provide emotionally engaging yet psychologically safer experiences than fully immersive VR. The following section highlights existing studies that have begun to explore AR and MR in schizophrenia education, and sets the foundation for the MR-based approach developed in this thesis.


\subsection{Augmented and Mixed Reality Applications}
\subsubsection{Key Studies}
\label{sec:keystudies}

An increasing number of studies are exploring the use of AR and MR in schizophrenia education. One early - and for this thesis very relevant - example is by Silva et al. (2017), who created a tool using AR to simulate psychotic symptoms. This system, developed with input from psychiatric professionals, was designed to help users — especially medical students — better understand schizophrenia and reduce stigma. The AR tool allowed users to interact with simulated symptoms in real time, providing a safe and controlled learning environment \cite{Silva2017}. To test the system, 21 medical students used AR glasses (HMZ-T2, Sony glasses \emph{ref?}) to experience the simulation. Afterward, they filled out questionnaires about their attitudes toward schizophrenia, how realistic they found the experience, and whether their views had changed. Students gave high ratings for the audio quality and educational value of the simulation. Many said it helped them better understand what psychotic experiences might feel like. However, some users also reported problems, such as discomfort from the equipment and difficulty focusing in the environment \cite{Silva2017}.

The simulations impact on empathy and stigma was measured using questionnaires before and after the experience. The results showed that students felt more empathy, expressed more concern for a fictional patient, and were more willing to help. However, there was also a small increase in stigma scores, showing that the results were complex. The study suggests that while AR can help increase empathy, future designs should focus on improving comfort and exploring long-term effects \cite{Silva2017}. It also recommends combining simulations with brief educational sessions on schizophrenia to deepen understanding \cite{Silva2017}. 

Another very relevant study by Skoy et al. created a simulation where users hear disturbing voices through headphones to better understand the kind of confusion and distraction that people with schizophrenia may deal with \cite{Skoy2016}. The simulation used Patricia Deegan's "Hearing Distressing Voices" audio track — based on her personal experience with schizophrenia — and was paired with practical tasks. Students completed these while listening to disturbing voices through headphones, mimicking real-life challenges. After the simulation, students took part in a debriefing and completed reflective writing. Results showed a significant increase in empathy scores.

A more recent project by Krogmeier et al. involved the development of \textit{Live-It}, an AR simulation that used the passthrough function of the Meta Quest 3 headset. This system simulated hallucinations and delusions in familiar places like living rooms or pharmacies. The design was based on real-life experiences from individuals with schizophrenia and was reviewed by neuropsychologists to ensure accuracy \cite{Krogmeier2024}. Participants in the study - mainly students and professionals in mental health — reported strong emotional reactions and said the simulation helped them better understand schizophrenia symptoms. One of the strengths of \textit{Live-It} was its ability to place symptoms into everyday situations, which made the experience feel more realistic and less overwhelming than fully immersive VR. For example, users heard voices that ranged from critical to supportive, reflecting the variety of hallucinations people might experience. The simulation also ended with hopeful messages, which helped balance the emotional impact. Overall, the study found that \textit{Live-It} increased empathy and encouraged participants to support individuals with schizophrenia. It showed that AR can be a powerful tool in mental health education, especially when it helps bridge the gap between theory and real-life experience \cite{Krogmeier2024}.

\subsubsection{Technical Advantages}

AR/MR simulations place symptoms in real-world settings, which can reduce user discomfort and improve relatability. These simulations tend to be less intense than full VR, making them more accessible to first-time users or those unfamiliar with immersive technology. One useful feature is passthrough, a technology that allows users to see their actual surroundings through cameras on the headset while digital content is overlaid on top. This helps users stay oriented and grounded in the real world while still experiencing simulated symptoms, which may enhance engagement while minimizing sensory overload. Features like passthrough may help enhancing empathy without overwhelming users \cite{Krogmeier2024, Silva2017, Lan2023}.

\section{Empathy in Healthcare Education}

\subsection{Definition of Empathy}
Empathy is a key part of good communication and care in healthcare. Many studies have shown that when healthcare professionals show empathy, patients are more satisfied, more likely to follow treatment plans, and often have better mental health outcomes \cite{Cunico2012, Olson1995, Ozcan2018}. In medical and nursing education, empathy is no longer seen as just a "soft skill." It is now treated as something important that can be taught and developed. Teaching empathy helps improve the way future professionals connect with patients and provide care \cite{Cunico2012}.

Empathy is usually described as having two main parts: \textit{cognitive empathy} and \textit{affective empathy}. Cognitive empathy is the ability to understand what someone else is thinking or feeling. Affective empathy means actually feeling or emotionally connecting with what the other person is going through \cite{Ventura2020, Martingano2021}. In healthcare, both types are important. Understanding a patient's perspective (cognitive empathy) helps with communication and decision-making, while emotional connection (affective empathy) helps build trust and stronger relationships \cite{Cunico2012, Ozcan2018}. 

Understanding this and training empathy helps doctors and nurses better understand their patients and respond in helpful and compassionate ways \cite{Ozcan2018, Olson1995}. However, research has shown that empathy can decrease during medical training. This might be because students are under pressure, focusing more on technical knowledge, or feeling emotionally drained \cite{Mattsson2024, Ozcan2018}. This decline in empathy can lead to negative outcomes for both patients and healthcare professionals. Patients may feel misunderstood or neglected, while healthcare providers may experience burnout and job dissatisfaction \cite{Mattsson2024, Cunico2012}. Therefore, it is crucial to find effective ways to teach and maintain empathy in medical education.

\subsection{Measuring Empathy}
\emph{maybe this is not really important here and should be placed in methodology section.}

Various instruments are used to measure these dimensions of empathy, including the Jefferson Scale of Empathy (JSE), which is being widely applied in medical education \cite{Alieldin2024}. This tool allows researchers to assess changes in empathy following interventions and distinguish between shifts in emotional versus cognitive components, which is also what I want to achieve in this thesis. In the context of this thesis, the JSE will be used to measure the impact of the MR simulation on medical students' empathy levels. The JSE is a validated instrument that has been widely used in medical education research and has demonstrated reliability and validity in assessing empathy in healthcare professionals \cite{Hojat2002}. By employing the JSE, this study aims to provide a comprehensive evaluation of the effectiveness of the MR simulation in enhancing both cognitive and affective empathy among medical students. In Chapter \emph{reference to results and analysis chapter}, we will discuss the results of the JSE and how they relate to the overall objectives of this thesis.

\subsection{Immersive Technologies and Empathy}

\subsubsection{Empathy Increase through Virtual Reality}

Virtual Reality (VR) has often been called the "ultimate empathy machine" because it can create powerful first-person experiences in fully immersive environments \cite{Milk2015}. Several studies support this idea, showing that VR can have a strong emotional effect on users.

VR is especially useful when it comes to helping people understand the experiences of stigmatized groups, such as individuals with schizophrenia \cite{Formosa2018, Marques2022, Mattsson2024}. These systems allow users to go through simulated versions of symptoms like hearing voices or feeling paranoid. By placing users in situations that reflect what it might be like to live with psychosis, these simulations aim to increase empathy and reduce negative attitudes. For example, Formosa et al. (2018) found that people who used a VR simulation of schizophrenia symptoms felt more empathy and showed less stigma afterwards compared to those who did not use the simulation \cite{Formosa2018}. A similar study by Hsia et al. (2022) showed that pharmacy students who experienced auditory hallucinations in VR also became more empathetic and less stigmatizing toward people with schizophrenia \cite{Hsia2022}. One crucial reason for this was that the students also heard from a guest speaker diagnosed with schizophrenia after they have experienced the simulation. This combined approach helps address one of the main concerns with simulations — that they can unintentionally increase social distance or reinforce stereotypes if not supported by real-life context. Including authentic human interaction can make the experience more meaningful and well-rounded. In this thesis we will also include a debriefing session after the simulation, where students can reflect on their experiences and discuss them with peers and instructors. This is important for helping students process what they have learned and apply it to real-life situations \cite{Hsia2022}.

% A recent mixed-methods study by Alieldin et al. (2024) tested the use of immersive virtual reality (IVR) to improve empathy among first-year medical students \cite{Alieldin2024}. The intervention involved a VR simulation called The Frank Lab, which placed students in the perspective of “Frank,” a socially isolated 72-year-old man. Students experienced the story from a first-person viewpoint and encountered situations related to grief, health decline, loneliness, and difficulties with daily life. The simulation included three storylines with interactive choices and was followed by a structured debriefing session. Empathy levels were measured before and after the training using the Jefferson Scale of Empathy (JSE). Students showed a statistically significant increase in empathy scores. In post-session interviews, participants described the simulation as powerful, emotional, and immersive — many said they felt “like Frank” or were “walking in his shoes.” These emotional responses were complemented by cognitive empathy, with students explaining how they better understood the struggles older adults face, especially with healthcare access and family relationships. The study also conducted follow-up interviews six months later. Students still remembered the experience vividly and reported applying what they had learned in patient interactions and standardized clinical scenarios. Importantly, the debriefing after the simulation was seen as a crucial component — helping students reflect, connect the experience to real-life practice, and deepen their learning. These results show how IVR can support both short-term empathy development and longer-term behavioral change, which supports the argument for immersive simulations in healthcare education \cite{Alieldin2024}.

% Formosa et al. (2018) tested a custom-built VR simulation designed to show what it feels like to experience positive symptoms of psychosis, such as auditory and visual hallucinations and paranoid thoughts \cite{Formosa2018}. Fifty participants — including students and people from the general public — completed a short immersive scenario inside a virtual house, where they heard disturbing voices, saw shadowy figures, and experienced delusions based on a fictional character’s background story.The researchers measured empathy, knowledge, and attitudes before and after the simulation. Results showed a significant increase in all three areas. Participants scored higher in understanding schizophrenia symptoms, felt more empathy, and reported more positive attitudes afterward. Interestingly, the simulation included no formal teaching — just the immersive experience itself — yet it still improved knowledge scores. This supports the idea that experiencing symptoms directly can be a powerful learning tool.The study also looked at how much participants “bought into” the simulation. People who felt the experience was realistic (fidelity) and useful (user buy-in) showed the biggest increase in empathy. This shows that emotional connection and meaningful design are key when using VR for education. The authors note that age influenced the experience — younger participants found it more immersive — but overall, the simulation was seen as a useful tool for teaching empathy and reducing stigma.

%A meta-analysis by Ventura et al. (2020) reviewed existing studies to see whether virtual reality (VR) can help increase empathy or perspective-taking \cite{Ventura2020}. The researchers analyzed data from seven studies and nine different participant groups. Overall, they found that VR had a moderate, statistically significant effect on perspective-taking — meaning users became better at imagining things from another person’s point of view. However, the impact on empathy itself was smaller and not statistically significant. This suggests that VR may be more effective at helping people understand others (cognitive empathy) than at creating deep emotional connection (affective empathy). The authors also noted that how the VR is designed matters: experiences that include a strong sense of “being there” (presence) or “being someone else” (embodiment) may be more effective. These findings support the idea that immersive tools like VR and MR can help build empathy — especially when they are designed thoughtfully and combined with reflection, as done in this thesis. 

The immersive tools discussed in this thesis are being tested in medical and nursing schools as a new way to teach empathy by letting students “step into the shoes” of patients \cite{Alieldin2024}. As mentioned earlier, this approach is becoming more popular in education and has shown promising results.

\subsubsection{Empathy Increase through Mixed Reality}

MR is gaining attention as a promising alternative to VR in empathy-focused education, particularly in mental health contexts. Unlike VR, MR allows users to remain partially connected to their physical surroundings while engaging with digitally simulated symptoms. This hybrid approach combines the immersive power of VR with the real-world anchoring of AR, helping to reduce sensory overload and making experiences more relatable and less overwhelming \cite{Zare-Bidaki2022}.

Studies by Silva et al. (2017) and Krogmeier et al. (2024), which were already discussed in detail in section~\ref{sec:keystudies} demonstrate the effectiveness of MR in increasing empathy and understanding toward individuals with schizophrenia. In both cases, simulations placed users in familiar environments while layering auditory and visual hallucinations over the reality. Participants reported strong emotional engagement and a clearer understanding of what it might be like to experience psychosis \cite{Silva2017, Krogmeier2024}. 

Together, these insights reinforce the central aim of this thesis: to evaluate MR as a balanced and effective tool for simulating psychotic experiences in medical education. By allowing users to engage empathetically with symptoms while staying cognitively oriented, MR may better support both affective and cognitive empathy development. Its ability to blend emotional immersion with realism makes it especially well-suited for sensitive topics like schizophrenia, where responsible storytelling and psychological safety are essential.


\subsection{Limitations in Empathy Training}

Martingano et al. (2021) reviewed 43 studies and found that while VR often enhances affective empathy, its effect on cognitive empathy is less consistent \cite{Martingano2021}. They argue that immersive experiences might reduce the user's need to mentally simulate another's perspective, as the simulation does that work for them. Without reflection or guided discussion, users may have strong emotional reactions but fail to develop deeper understanding.

Similarly, Rueda and Lara (2020) caution against relying on emotional responses alone. They call for "reason-guided empathy," which integrates critical thinking and ethical reflection into simulation-based learning \cite{Rueda2020}. Without this, empathy may be short-lived or biased.

The findings also show that more expensive or immersive setups do not necessarily yield better outcomes. Thoughtful design and context are incredibly important. Many VR simulations rely heavily on dramatic intensity, which can restrict the ability of the user to reflect or exercise perspective-taking — the cognitive process of imagining the world from another person’s viewpoint, which is essential for developing empathy and reducing bias \cite{Mattsson2024}. This limitation further supports the use of MR paired with preparation and debriefing, as adopted in this thesis.

Ozcan et al. tracked empathy development in nursing students over four years. While communication skills improved, emotional empathy declined—likely due to burnout or emotional distancing \cite{Ozcan2018}. This underlines the importance of designing empathy training that includes emotional support and reflection. The MR simulation in this thesis builds on that principle.

Finally, as mentioned repeatedly, ethical concerns remain. VR simulations can unintentionally reinforce negative stereotypes if not carefully framed. Being incredibly affected by something, without deeper context, may lead to bias or stigma \cite{Rueda2020}. MR used in the real world, along with structured pre- and post-simulation activities, is intended to reduce this risk. The approach in this thesis prioritizes both emotional resonance and cognitive clarity to improve thoughtful empathy in clinical learners.


\section{Simulation Design Considerations}
\subsection{Empathy and Usability}
Immersive simulations offer powerful opportunities to increase empathy in medical education, particularly for conditions like schizophrenia. However, designing effective simulations requires careful attention to realism, emotional impact, and usability.

Marques et al. compared a VR simulation of psychosis with a standard 2D video and found that the VR group experienced greater gains in cognitive empathy and held more positive attitudes toward individuals with schizophrenia. However, the study also noted several limitations: it lacked a control group and did not measure perceived immersion — a key factor in empathy development. Some participants also struggled with unfamiliarity with the technology \cite{Marques2022}.

Similarly, Zare-Bidaki et al. found that VR simulations of psychosis led to higher empathy and stigma reduction compared to traditional patient visits. However, they emphasized that simulations should supplement—not replace—direct human interactions. Authentic contact provides depth, variability, and personal meaning, which simulations alone cannot replicate \cite{Zare-Bidaki2022, Hsia2022}.

Both studies emphasize that simulations must balance engagement and emotional intensity without overwhelming participants. Overly dramatic portrayals of symptoms — such as frightening hallucinations or paranoia — can trigger distress, increase social distance, or reinforce harmful stereotypes if not properly contextualized \cite{Ando2011, Chaffin2013, Zare-Bidaki2022}.

To reduce this risk, Zare-Bidaki et al. recommend using calm, familiar environments and grounding simulations in lived experience. They also suggest that AR or MR, which preserve awareness of the real world, may help avoid overstimulation while still enabling emotional immersion \cite{Zare-Bidaki2022}. This aligns with the approach taken in this thesis, which uses MR to simulate symptoms in relatable real-world contexts. The use of passthrough features allows participants to remain anchored while interacting with hallucination overlays, aiming to foster empathy without sensory overload.

\subsection{Ethical Challenges}

While immersive simulations hold great promise for enhancing empathy, they also raise important ethical and psychological concerns—particularly in the context of mental health education as seen in the previous sections. Many studies suggest that emotional impact alone does not guarantee positive attitudinal change and may, in some cases, amplify discomfort or misunderstanding \cite{Ando2011}.

These findings highlight the critical importance of proper preparation and debriefing. Without guided reflection, users may interpret psychotic symptoms in simplistic or fear-based ways, reinforcing stereotypes about schizophrenia. Ando et al. and Rueda and Lara both advocate for what they call a \textit{reason-guided empathy}, a model in which emotional engagement is supported by ethical reflection and cognitive understanding. This approach encourages users not only to feel compassion but also to think critically about the lived experience of mental illness \cite{Ando2011, Rueda2020}.

Another important ethical issue has to do with how the simulation is designed. Using very realistic effects—like intense visuals, surround sound, and dramatic symptoms—can make the experience feel more lifelike. But for some users, especially those not used to immersive technology, this can be overwhelming. Also, trying to show a “typical” psychotic episode can be problematic, since symptoms vary a lot from person to person. This could lead to a simplified or even misleading picture of what schizophrenia is really like \cite{Zare-Bidaki2022}.

It iss also essential to think about how the story behind the symptoms is presented. If the simulation focuses only on fear or confusion without any background or explanation, it might unintentionally make people with schizophrenia seem dangerous or unstable. This can reinforce negative stereotypes. Rueda and Lara warn that mental health simulations need to be told in a responsible way—showing the human side of the experience, not just the symptoms \cite{Rueda2020}.

In conclusion, When used alongside proper educational materials and opportunities to reflect on the experience, MR can help build deeper, more respectful empathy. This is a key part of the design approach taken in this thesis.


\subsection{Simulation Design Strategy}
To address the challenges mentioned above, this thesis adopts a design strategy that:

\begin{itemize}
    \item Uses Mixed Reality to simulate schizophrenia symptoms in familiar environments, allowing users to remain grounded in reality
    \item Tests the simulation on medical students which already have had a preparatory educational session to provide context and understanding of schizophrenia, reducing the risk of reinforcing stigma
    \item Includes a debriefing session to help with reflection, discussion, and ethical understanding of the experience
    \item Measures perceived immersion and empathy outcomes to evaluate the impact of the simulation on students' attitudes and understanding
    \item Uses a combination of auditory and visual hallucinations to create a layered experience that reflects the complexity of real-life symptoms
    \item Uses a gradual increase in emotional intensity, allowing users to acclimate to the experience without overwhelming them
    \item Engages students in a reflective process that encourages them to connect their experiences to real-life clinical practice and patient interactions
\end{itemize}

By doing so, this approach aims to increase both affective and cognitive empathy in medical students — helping them not only to feel what patients go through, but also to understand their experiences within a respectful and informed framework.



\begin{landscape}
    \scriptsize
    \begin{longtable}{|p{2.8cm}|p{0.6cm}|p{1.8cm}|p{1.2cm}|p{1.2cm}|p{2cm}|p{1.2cm}|p{1.2cm}|p{1.2cm}|p{3cm}|}
    \caption{Overview of studies used for this thesis} \label{tab:studies} \\
    \hline
    Title & Year & Study Design & Tools Used (VR/AR/MR) & Target Group & Symptom Experience & Empathy or Stigma & Cognitive Empathy Increased & Affective Empathy Increased & Main Results \\
    \hline
    \endfirsthead
    \hline
    Title & Year & Study Design & Tools Used (VR/AR/MR) & Target Group & Symptom Experience & Empathy or Stigma & Cognitive Empathy Increased & Affective Empathy Increased & Main Results \\
    \hline
    \endhead
    
    Developing empathy in nursing students: a cohort longitudinal study & 2012 & Cohort longitudinal & None (Traditional education methods) & Nursing students & General emotional and communication contexts & Empathy & Yes & Yes & Empathy improved significantly in women through targeted training; results less clear for men \\
    \hline
    Impact of a Virtual Reality-Based Simulation on Empathy and Attitudes Toward Schizophrenia & 2022 & Quasi-experimental & VR & Health students & Simulated psychotic symptoms & Both & Yes & Possibly & VR more effective than 2D video in enhancing empathy and reducing stigma \\
    \hline
    Empathic Mixed Reality: Sharing What You Feel and Interacting with What You See & 2017 & Experimental (early studies) & MR (AR + VR) & General users (not specified) & Emotion sharing, collaboration & Empathy & Possibly & Yes & MR enabled physiological and emotional data sharing; promising for collaborative empathy \\
    \hline
    Nursing Students' Experiences of Empathy in a Virtual Reality Simulation Game & 2024 & Descriptive qualitative & VR & Nursing students & Virtual patient care & Empathy & Yes & Yes & VR helped students experience and express empathy effectively \\
    \hline
    Virtual Reality as a Medium to Elicit Empathy: A Meta-Analysis & 2020 & Meta-analysis & VR & Various populations & Multiple contexts & Empathy & Yes & Unclear & Perspective-taking improved; general empathy results were mixed \\
    \hline
    Improving Empathy in Nursing Students: A Comparative Longitudinal Study of Two Curricula & 2018 & Comparative longitudinal & None (Traditional vs. integrated curriculum) & Nursing students & General emotional and clinical context & Empathy & Yes & Decreased over time & Integrated curriculum more effective; empathic skills improved but tendency declined \\
    \hline
    Relationships Between Nurse\textendash Expressed Empathy, Patient\textendash Perceived Empathy and Patient Distress & 1995 & Correlational study & None (standard clinical practice) & Nurses and patients & Real-life distress in hospital settings & Empathy & Not applicable & Not applicable & Nurse-expressed empathy positively correlated with perceived empathy; reduced patient distress \\
    \hline
    Testing the efficacy of a virtual reality based simulation in enhancing users' knowledge, attitudes and empathy relating to psychosis & 2018 & Experimental pre-post & VR & General public, psychology students & Simulated psychotic symptoms & Both & Yes & Yes & VR simulation significantly increased empathy, knowledge, and improved attitudes \\
    \hline
    Virtual Reality and Empathy Enhancement: Ethical Aspects & 2020 & Theoretical/Review & VR & General (conceptual discussion) & Not specific (broad scenarios) & Empathy & Possibly & Possibly & Explores philosophical and ethical aspects; emphasizes reason-guided empathy over immersive emotion \\
    \hline
    Effectiveness of immersive virtual reality in teaching empathy to medical students & 2024 & Mixed methods (pre-post + interviews) & VR & Medical students & Social isolation in older adults & Empathy & Yes & Yes & Empathy significantly increased post-training; immersion and embodiment were key factors \\
    \hline
    VR Improves Emotional but Not Cognitive Empathy: A Meta-Analysis & 2021 & Meta-analysis & VR & General population & Various contexts & Empathy & No & Yes & VR improved emotional empathy but not cognitive empathy; not more effective than low-tech methods \\
    \hline
    Use of an Auditory Hallucination Simulation to Increase Student Pharmacist Empathy & 2016 & Pre-post experimental & Audio simulation & Pharmacy students & Auditory hallucinations & Empathy & Not measured & Yes & Empathy increased; students reported distraction and frustration during task \\
    \hline
    Reducing the Schizophrenia Stigma: A New Approach Based on Augmented Reality & 2017 & Quasi-experimental & AR & Medical students & Psychotic symptoms simulation & Stigma & Not measured & Not measured & AR experience reduced stigma and improved understanding of schizophrenic symptoms \\
    \hline
    Leveraging AR for Understanding Schizophrenia & 2024 & Thematic evaluation (qualitative) & AR & Healthcare students, experts & Hallucinations, delusions, disorganized behavior & Stigma & Possibly & Possibly & Participants better understood schizophrenia; highlighted as an educational tool \\
    \hline
    The Virtual Doppelganger: Effects of a Virtual Reality Simulator on Perceptions of Schizophrenia & 2010 & Between-subjects experiment (4 conditions) & VR & General public & Schizophrenia symptoms & Both & Yes (in combo with empathy set) & Yes & Empathy + VR condition most effective; VR-only increased social distance \\
    \hline
    Immersive VR Applications in Schizophrenia Spectrum Therapy: A Systematic Review & 2020 & Systematic review & VR & Patients with schizophrenia spectrum disorders & Delusions, hallucinations, cognitive/social issues & Empathy (implied), Therapy & Not directly measured & Not directly measured & VR showed promising results for therapy; safe and well tolerated \\
    \hline
    Efficacy of Immersive XR Interventions on Symptoms of Schizophrenia Spectrum Disorders & 2023 & Systematic review & XR (VR) & Patients with schizophrenia & Various psychotic symptoms & Empathy (secondary), Therapy & Not focus & Not focus & VR effective across symptom domains; no AR studies found \\
    \hline
    Impact of an Auditory Hallucination Simulation Coupled with a Speaker Diagnosed with Schizophrenia & 2022 & Pre-post with speaker intervention & Audio simulation & Pharmacy students & Auditory hallucinations & Stigma & Not focus & Not focus & Stigma reduced significantly, especially in attitudes and disclosure openness \\
    \hline
    Representing Mental Disorders with Virtual Reality: Goliath & 2023 & Case study analysis (artistic VR) & VR & General public & Narrative VR of schizophrenia & Empathy & Yes & Yes & Focused on ethical, artistic VR design for empathy through embodiment \\
    \hline
    The Simulation of Hallucinations to Reduce the Stigma of Schizophrenia: A Systematic Review & 2011 & Systematic review & Simulation (audio/VR) & Mixed (students, general public) & Hallucination simulation & Stigma & No & Yes & Increased empathy, but also social distance; ethical considerations advised \\
    \hline
    Creating Empathy Through Use of a Hearing Voices Simulation & 2013 & Mixed methods (pre-post and reflection) & Audio simulation & Psychiatric nursing students & Auditory hallucinations & Empathy & Not measured & Yes & Empathy significantly increased; students reported transformation in attitude and care approach \\
    \hline
    Out of Touch with Reality? Social Perception in First-Episode Schizophrenia & 2013 & fMRI observational study & None (neuroimaging) & Schizophrenia patients & Tactile and social perception stimuli & Empathy & Impaired (linked to self-other confusion) & Not measured & Impaired neural mechanisms for social touch perception; linked to empathy deficits \\
    \hline
    Immersive Simulation of Schizophrenia & 2023 & Development and evaluation project & VR & General public / students & Visual and auditory hallucinations & Stigma & Possibly & Possibly & VR simulation aimed to reduce stigma; immersive experience showed promise for education \\
    \hline
    Learning by Doing: Educational VR for Care of Schizophrenic Patients & 2020 & Design and usability study & VR (360 video, HMD) & Nursing students & Various schizophrenia symptoms (hallucinations, delusions) & Empathy & Possibly & Yes & Participants reported increased empathy and engagement; useful educational platform \\
    \hline
    Evaluating VR Simulation of Psychosis on Stigma, Empathy, and Knowledge & 2022 & Controlled experimental & VR & Medical students & Psychotic symptoms & Both & Yes & Yes & VR significantly more effective than ward visits at increasing empathy and reducing stigma \\
    \hline
    Usability of Mental Illness Simulation via Immersive VR & 2020 & Mixed methods usability study & VR & Nursing students & Schizophrenia symptoms & Empathy & Possibly & Yes & Students found simulation realistic and engaging; suggested for broader use in nursing education \\
    \hline
    Visual Hallucinations in Psychosis & 2019 & Clinical observational study & None & Psychosis patients & Visual hallucinations (VH) & Empathy (implied) & Not measured & Not measured & VH are diverse and vivid; associated with reduced insight and fear; linked to stigma and distress \\
    \hline
    Visual Distortions and Hallucinations in Schizophrenia: An Update & 2021 & Literature review & None & Schizophrenia patients & Visual hallucinations and distortions & Empathy (conceptual) & Not directly assessed & Not directly assessed & Explores mechanisms and clinical impact of visual symptoms; calls for targeted interventions \\
    \hline
    \end{longtable}
    \end{landscape}